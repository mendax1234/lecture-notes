\documentclass[a4paper]{article}
\usepackage[margin=1in]{geometry} % Optional: Adjust margins
\usepackage{graphicx} % Required for inserting images
\usepackage{amsmath,amsthm,amsfonts,amssymb}

\title{Classic Examples for Differential Equations}
\author{Wenbo Zhu}
\date{November 2024}

\begin{document}

\maketitle

\section{Introduction to Differential Equations}
\subsection{The method of Separation of variables}
\begin{enumerate}
    \item (\textbf{Linear change of variable}) Solve the differential equation
    \begin{equation}
        \frac{dy}{dx}=\frac{1-2y-4x}{1+y+2x}
    \end{equation}
    \textbf{Solution.} Observe that we may rewrite the differential equation as
    \begin{equation*}
        \frac{dy}{dx}=\frac{1-2(y+2x)}{1+(y+2x)}
    \end{equation*}
    We employ a \textit{linear change of variable}: let
    \begin{equation*}
        u=y+2x\Rightarrow \frac{du}{dx}=\frac{dy}{dx}+2 \text{ or, }u'=y'+2
    \end{equation*}
    Thus, our differential equation becomes
    \begin{equation*}
        \frac{du}{dx}=\frac{1-2u}{1+u}+2\Rightarrow\frac{du}{dx}=\frac{3}{1+u} \text{ or, }u'=\frac{3}{1+u}
    \end{equation*}
    This is now a separable equation!
    \begin{equation*}
        \int(1+u)~du=\int3~dx\Rightarrow u+\frac{u^2}{2}=3x+c
    \end{equation*}
    Since $u=y+2x$, we thus have the general solution\footnote{This is in \textit{implicit form}. The general solution can be in implicit form.}
    \begin{equation*}
        y+2x+\frac{(y+2x)^2}{2}=3x+c
    \end{equation*}
    \newpage
    \item (\textbf{Fraction change of variable}) Solve the differential equation
    \begin{equation}
        2xy\frac{dy}{dx}-y^2+x^2=0
    \end{equation}
    \textbf{Solution.} Observe that we may rewrite the differential equation as
    \begin{equation*}
        \frac{dy}{dx}=\frac{y^2-x^2}{2xy}=\frac{1}{2}(\frac{y}{x}-\frac{x}{y}).
    \end{equation*}
    Consider the substitution $u=\frac{y}{x}$, or $y=xu$, such that $y'=u+xu'$\footnote{This is done by \textit{product rule}}. Substituting this into the differential equation yields
    \begin{equation*}
        u+x\frac{du}{dx}=\frac{1}{2}(u-\frac{1}{u})\Rightarrow x\frac{du}{dx}=-\frac{1}{2}(u+\frac{1}{u})
    \end{equation*}
    Observe that this is now a separable differential equation!
    \begin{equation*}
        \int\frac{2u}{u^2+1}~du=\int-\frac{1}{x}~dx\Rightarrow \ln(u^2+1)=-\ln x+c
    \end{equation*}
    Since $u=\frac{y}{x}$, we thus obtain the general solution\footnote{Here the constant $A=e^c$, which will be determined by the initial condition}
    \begin{equation*}
        ln(\frac{y^2}{x}+x)=c\Rightarrow\frac{y^2}{x}+x=A.
    \end{equation*}
    \item (\textbf{Special substitution}) Solve the following differential equation with given initial conditions
    \begin{equation}
        y'y''=2\text{, with }y(0)=1\text{ and }y'(0)=2.
    \end{equation}
    \textbf{Solution.} To reduce the order of the differential equation, let $u=y'$ and we will have $u'=y''$. Substituting this into the differential equation yields
    \begin{equation*}
        u\frac{du}{dx}=2
    \end{equation*}
    Observe that this is now a separable differential equation!
    \begin{equation*}
        \int u~du=\int 2~dx \Rightarrow \frac{u^2}{2}=2x+c
    \end{equation*}
    Since $u=y'$, the substitution yields
    \begin{equation*}
        (y')^2=4x+2c
    \end{equation*}
    Since $y'(0)=2$, then $2c=4$. We can now directly integrate our expression for $y'$ to find $y$
    \begin{equation*}
        y'=(4x+4)^{\frac{1}{2}}=2(x+1)^{\frac{1}{2}}\Rightarrow y=\frac{4}{3}(x+1)^{\frac{3}{2}}+D
    \end{equation*}
    Since $y(0)=1$, after substitution, we can get
    \begin{equation*}
        1=\frac{4}{3}(0+1)^{\frac{3}{2}}+D\Rightarrow D=-\frac{1}{3}
    \end{equation*}
    Therefore, the solution to the initial value problem is given by
    \begin{equation*}
        y=\frac{4}{3}(x+1)^{\frac{3}{2}}-\frac{1}{3}
    \end{equation*}
\end{enumerate}
\subsection{Population Model}
\begin{enumerate}
    \item (\textbf{Application of Equilibrium Solutions}) The population of a certain species of bugs behaves according to the Verhulst (logistic) model, and the formula is given below:
    \begin{equation}
        \frac{dy}{dt}=[k(1-\frac{y}{y_\infty})]y
    \end{equation}
    where $k=-15, y_{\infty}\approx375.75$. Now, what is the maximum number of bugs you can put to death per day without causing the population to die out? \newline
    \textbf{Solution.} We may now modify the Verhulst equation to account for the harvesting of bugs. Suppose $E$ bugs are put to death per day. Then
    \begin{equation*}
        \frac{dy}{dt}=[k(1-\frac{y}{y_{\infty}})]y-E=-\frac{k}{y_\infty}y^2+ky-E
    \end{equation*}
    where $k=1.5$ and $y_\infty\approx375.75$. Now, the equilibrium solutions are
    \begin{equation*}
        y=\frac{-k\pm\sqrt{k^2-\frac{4kE}{y_\infty}}}{-\frac{2k}{y_\infty}}=\frac{1.5\mp\sqrt{2.25-\frac{6E}{375.75}}}{\frac{3}{375.75}}
    \end{equation*}
    The population will die out when $E$ is chosen such that there are no equilibria. In this case, $\frac{dy}{dt}$ will be \textbf{negative} and any solution $y(t)$ will always be decreasing. This occurs when,
    \begin{equation*}
        k^2-\frac{4kE}{y_\infty}<0\Rightarrow E>\frac{375.75}{6}\times2.25=140.91\dots
    \end{equation*}
    Hence, we can only kill as much as 140 bugs per day.
\end{enumerate}
\newpage
\section{Linear Differential Equations}
\subsection{The method of Integrating Factors}
\begin{enumerate}
    \item Solve the following first-order linear differential equation
    \begin{equation}
        y'-(1+3x^{-1})y=x+2
    \end{equation}
    \textbf{Solution.} The equation is in standard form. The integrating factor is $e^{\int p(x)~dx}$, where $p(x)=-1-3x^{-1}$. Thus,
    \begin{equation*}
        \int p(x)~dx=-\int 1+3x^{-1}~dx=-x-3\ln x
    \end{equation*}
    and we have the integrating factor
    \begin{equation*}
        e^{\int p(x)~dx}=e^{-x-3\ln x}=x^{-3}e^{-x}
    \end{equation*}
    Multiplying this factor to both sides of the equation,
    \begin{align*}
        (x^{-3}e^{-x})[y'-(1-\frac{3}{x})y]&=(x^{-3}e^{-x})[x+2] \\
        (x^{-3}e^{-x}y)'&=x^{-2}e^{-x}+2x^{-3}e^{-x}
    \end{align*}
    Integrating both sides of the equation yields\footnote{Here, integrate R.H.S w.r.t $x$. L.H.S not sure}
    \begin{equation*}
        x^{-3}e^{-x}y+c=\int x^{-2}e^{-x}~dx+2\int x^{-3}e^{-x}~dx
    \end{equation*}
    To solve the integral in the second term, we integrate by parts: let $u=e^{-x},du=-e^{-x}~dx$, and $dv=x^{-3}~dx, v=\frac{-x^{-2}}{2}$:\footnote{Note that $\int e^{-x}x^{-2}~dx$ is unsolvable using my limited knowledge.} 
    \begin{equation*}
        2\int x^{-3}e^{-x}~dx=-e^{-x}x^{-2}-\int x^{-2}e^{-x}~dx
    \end{equation*}
    Thus, we have\footnote{Here we cancel off the term $\int x^{-2}e^{-x}~dx$}
    \begin{equation*}
        x^{-3}e^{-x}y+c=-e^{-x}x^{-2}
    \end{equation*}
\end{enumerate}
\subsection{Bernoulli differential equation}
\begin{enumerate}
    \item (\textbf{Normal Substitution}) Find the general solution to the following differential equation
    \begin{equation} \label{eq:bernoulli-1}
        2xyy'+(x-1)y^2=x^2e^x
    \end{equation} 
    \textbf{Solution.} Expressing the differential equation into Bernoulli form to find our $v$, we have
    \begin{equation*}
        2xy'+(x-1)y=x^2e^xy^{-1}
    \end{equation*}
    Now, we can let $v=y^2$, and $v'=2yy'$, substituting them into (\ref{eq:bernoulli-1}) yields\footnote{Afterwards, we should write the equation in $v$ in the linear form, which is the coefficient of $v'$ must be 1}
    \begin{equation*}
        xv'+(x-1)v=x^2e^x \Rightarrow v'+(1-x^{-1})v=xe^x
    \end{equation*}
    We have the integrating factor $e^{\int 1-\frac{1}{x}~dx}=e^{x-\ln x}=e^xx^{-1}$. Multiplying this to both sides of the equation,
    \begin{align*}
        (e^xx^{-1})[v'+(1-x^{-1})v]&=(\frac{e^x}{x})xe^x \\
        (e^xx^{-1}v)'&=e^{2x}
    \end{align*}
    Integrating both sides, we have
    \begin{equation*}
        \frac{e^x}{x}y^2=\frac{1}{2}e^{2x}+c
    \end{equation*}
    \item (\textbf{Special Substitution}) Find the general solution to the following differential equation
    \begin{equation}
        x^2+\sin y=y'\cos y
    \end{equation}
    \textbf{Solution.} This is a Bernoulli equation: letting $v=\sin y,v'=y'\cos y$, the differential equation becomes
    \begin{equation*}
        x^2+v=v'\Rightarrow v'-v=x^2
    \end{equation*}
    We have the integrating factor $e^{\int-1~dx}=e^{-x}$. Multiplying this to both sides of the equation,
    \begin{equation*}
        e^{-x}[v'-v]=e^{-x}[x^2]\Rightarrow (e^{-x}v)'=e^{-x}x^2
    \end{equation*}
    We integrate the right-hand side by parts:
    \begin{equation*}
        \int e^{-x}x^2~dx=-x^2e^{-x}+2\int e^{-x}x~dx=-x^2e^{-x}+2(-xe^{-x}-e^{-x})+c
    \end{equation*}
    Substituting back $v=\sin y$, we find that
    \begin{equation*}
        e^{-x}\sin y=-x^2e^{-x}-2xe^{-x}-2e^{-x}+c
    \end{equation*}    
\end{enumerate}
\newpage
\section{The Harmonic Oscillator}
\subsection{The method of undetermined coefficients}
\begin{enumerate}
    \item (\textbf{Don't have Trigonometric Terms}) Find a particular solution to
    \begin{equation}
        y'''-2y''+y'=2x
    \end{equation}
    \textbf{Solution.} Firstly, the characteristic equation is $\lambda^3-2\lambda^2+\lambda=\lambda(\lambda-1)^2=0$, so
    \begin{equation*}
        y_h=c_1+c_2e^x+c_3xe^x
    \end{equation*}
    To find the particular solution, try $Ax+B$. Since a constant term appears in $y_h$ ($c_1$), we modify this trial solution: instead, try $y=Ax^2+Bx$. Then,
    \begin{equation*}
        y'=2Ax+b,y''=2A,y'''=0
    \end{equation*}
    Plugging this into the equation yields
    \begin{equation*}
        0-4A+2Ax+B=2x\Rightarrow 2Ax+(-4A+B)=2x+0
    \end{equation*}
    Comparing coefficients, we have $A=1$ and $B=4A=4$. Thus,
    \begin{equation*}
        y_p=x^2+4x
    \end{equation*}
    \item (\textbf{Have Trigonometric Terms}) Find a particular solution to
    \begin{equation}
        y''+2y'+3y=34e^x\cos 2x
    \end{equation}
    \textbf{Solution.} First, we observe that $f(x)=34e^x\cos 2x$ is equal to
    \begin{equation*}
        \mathfrak{Re}(34e^xe^{i2x})=\mathfrak{Re}(34e^{x(1+2i)})
    \end{equation*}
    Thus, we can equivalently solve for
    \begin{equation*}
        y''+2y'+3y=34e^{x(1+2i)}
    \end{equation*}
    noting that we only want the real part of this solution. We try
    \begin{align*}
        y_p&=Ae^{x(1+2i)}, \\
        y_p'&=A(1+2i)e^{x(1+2i)}, \\
        y_p''&=A(1+2i)^2e^{x(1+2i)}=A(-3+4i)e^{x(1+2i)},
    \end{align*}
    Substituting this into the equation, we have
    \begin{equation*}
        A(-3+4i)e^{x(1+2i)}+2A(1+2i)e^{x(1+2i)}+3Ae^{x(1+2i)}=34e^{x(1+2i)}
    \end{equation*}
    and $A(2+8i)=34$. Simplifying this, we find
    \begin{equation}
        A=\frac{34}{2+8i}=\frac{17}{1+4i}\cdot\frac{1-4i}{1-4i}=\frac{17-68i}{17}=1-4i
    \end{equation}
    So,
    \begin{align*}
        y_p&=(1-4i)e^{x(1+2i)}=e^x(1-4i)(\cos 2x+i\sin 2x) \\
        &=e^x(\cos 2x+4\sin 2x+i(-4\cos 2x+\sin 2x))
    \end{align*}
    Thus, a particular solution is given by
    \begin{equation*}
        \mathfrak{Re}(y_p)=e^x(\cos 2x+4\sin 2x)
    \end{equation*}
\end{enumerate}
\newpage
\section{The Laplace Transform}
\subsection{Unit Step Functions and Dirac Delta Functions}
\begin{enumerate}
    \item (\textbf{Application of Laplace transform on Unit Step Functions}) Do the Laplace Transform on the following unit step function
    \begin{equation}
        \frac{t^2}{2}\cdot u(t-1)
    \end{equation}
    \textbf{Solution.} We apply the Second Shifting Theorem\footnote{$\mathcal{L}[f(t-c)u(t-c)]=e^{-sc}\mathcal{L}[f(t)]$ or $\mathcal{L}[f(t)u(t-c)]=e^{-sc}\mathcal{L}[f(t+c)]$}
    \begin{align*}
        \mathcal{L}[\frac{t^2}{2}\cdot u(t-1)]&=\frac{1}{2}e^{-s}\mathcal{L}[(t+1)^2]\\
        &=\frac{e^{-s}}{2}\mathcal{L}[t^2+2t+1]\\
        &=\frac{e^{-s}}{2}(\frac{2!}{s^3}+\frac{2}{s^2}+\frac{1}{s})
    \end{align*}
    \item (\textbf{Application of the Inverse Laplace Transform}) Solve the following initial value problem
    \begin{equation}
        y'=tu(t-2),y(0)=4
    \end{equation}
    \textbf{Solution.} Take the Laplace Transform of both sides yields
    \begin{equation*}
        s\mathcal{L}[y]-y(0)=\mathcal{L}[tu(t-2)]
    \end{equation*}
    To compute the right-hand side, we apply the Second-Shifting Theorem to $f(t)=t$
    \begin{equation*}
        \mathcal{L}[tu(t-2)]=e^{-2s}\mathcal{L}[t+2]=e^{-2s}(\frac{1}{s^2}+\frac{2}{s})
    \end{equation*}
    Thus, we have
    \begin{equation*}
        s\mathcal{L}[y]-4=e^{-2s}(\frac{1}{s^2}+\frac{2}{s})\Rightarrow \mathcal{L}[y]=e^{-2s}(\frac{1}{s^3}+\frac{2}{s^2})+\frac{4}{s}
    \end{equation*}
    We now take the inverse Laplace transform of both sides. We evaluate the first term on the right: by the Second Shifting Theorem,
    \begin{equation*}
        \mathcal{L}^{-1}[e^{-2s}(\frac{1}{s^3}+\frac{2}{s^2})]=f(t-2)u(t-2)\text{, where }\mathcal{L}[f(t)]=\frac{1}{s^3}+\frac{2}{s^2}
    \end{equation*}
    Since
    \begin{equation*}
        f(t)=\mathcal{L}^{-1}[F(s)]=\mathcal{L}^{-1}[\frac{1}{s^3}]+\mathcal{L}^{-1}[\frac{2}{s^2}]=\frac{t^2}{2}+2t
    \end{equation*}
    we find that,
    \begin{equation*}
        f(t-2)=\frac{(t-2)^2}{2}+2(t-2)=\frac{(t-2)(t+2)}{2}=\frac{t^2-4}{2}
    \end{equation*}
    Thus,
    \begin{equation*}
        \mathcal{L}^{-1}[e^{-2s}\cdot \frac{1+2s}{s^3}]=(\frac{t^2}{2}-2)\cdot u(t-2)
    \end{equation*}
    Thus, the solution to the initial value problem is
    \begin{equation*}
        y(t)=(\frac{t^2}{2}-2)\cdot u(t-2)+4
    \end{equation*}
\end{enumerate}
\newpage
\section{Partial Differential Equations}
\subsection{The method of Separation of variables}
\begin{enumerate}
    \item (PDE becomes fake ODE) Find a particular solution $u(x,y)$ to the partial differential equation
    \begin{equation*}
        u_{xx}-2u_x-3u=0
    \end{equation*}
    given the initial conditions $u(0,y)=2y$ and $u_x=(0,y)=2y-4\sin y$ \textit{Hint: The partial differential equation only involves derivatives in $x$}
    
    \textbf{Solution.}  We treat the partial differential equation as ordinary differential equation in $x$. Then, we have a second-order homogeneous linear differential equation:
    \begin{equation*}
        u''-2u'-3u=0\Rightarrow \lambda^2-2\lambda-3=(\lambda+1)(\lambda-3)=0
    \end{equation*}
    The partial differential equation thus has the general solution
    \begin{equation*}
        u(x,y)=f(y)e^{-x}+g(y)e^{3x}
    \end{equation*}
    where $f$ and $g$ are functions of $y$ (i.e., constants with respect to $x$). By the initial conditions, we have
    \begin{equation*}
        u(0,y)=f(y)+g(y)=2y
    \end{equation*}
    Likewise, taking the partial derivative of $u$ with respect to $x$,
    \begin{equation*}
        u_x=-fe^{-x}+3ge^{3x}\Rightarrow u_x(0,y)=-f(y)+3g(y)=2y-4\sin y
    \end{equation*}
    We thus require that
    \begin{equation*}
        \begin{cases}
            f+g&=2y\\
            -f+3g&=2y-4\sin y
        \end{cases}
    \end{equation*}
    Solve it, we have
    \begin{equation*}
        \begin{cases}
            f(y)&=y+\sin y \\
            g(y)&=y-\sin y
        \end{cases}
    \end{equation*}
    Hence, the desired solution to the partial differential equation is
    \begin{equation*}
        u(x,y)=(y+\sin y)e^{-x}+(y-\sin y)e^{3x}
    \end{equation*}
\end{enumerate}
\subsection{The Heat Equation}
\begin{enumerate}
    \item (\textbf{Superposition Principle}) Consider the following partial differential equation:
    \begin{equation}
        u_t=2u_{xx},0\leq x\leq 3, t>0
    \end{equation}
    Solve the above equation with the given boundary and initial conditions
    \begin{equation*}
        u(0,t)=u(3,t)=0,u(x,0)=\frac{5}{8}\sin \pi x-\frac{5}{16}\sin 3\pi x+\frac{1}{16}\sin 5\pi x
    \end{equation*}
    \textbf{Solution.} This is simply the heat equation with $c^2=2,l=3$. Hence, our solutions will have the form
    \begin{equation*}
        u(x,t)=\beta_ne^{-2n^2\pi^2t/9}\sin(\frac{n\pi}{3}x)
    \end{equation*}
    By the initial condition $u(x,0)$, we have at $t=0$
    \begin{equation*}
        \beta_n\sin(\frac{n\pi}{3}x)=\frac{5}{8}\sin \pi x-\frac{5}{16}\sin 3\pi x+\frac{1}{16}\sin 5\pi x
    \end{equation*}
    By the Fundamental Theorem of Superposition, we can simply combine the solutions that produces each term above. Comparing coefficients we have $n=3,n=9,$ and $n=15$, with $\beta_3=5/8,\beta_9=-5/16$, and $\beta_{15}=1/16$, respectively. Hence, we have the particular solution
    \begin{equation*}
        u(x,t)=\frac{5}{8}e^{-2\pi^2t}\sin(\pi x)-\frac{5}{16}e^{-18\pi^2t}\sin(3\pi x)+\frac{1}{16}e^{-50\pi^2t}\sin(5\pi x)
    \end{equation*}
\end{enumerate}
\end{document}
