\documentclass[10pt, landscape]{article}
\usepackage[scaled=0.92]{helvet}
\usepackage{calc}
\usepackage{multicol}
\usepackage{ifthen}
\usepackage[a4paper,margin=3mm,landscape]{geometry}
\usepackage{amsmath,amsthm,amsfonts,amssymb}
\usepackage{color,graphicx,overpic}
\usepackage{hyperref}
\usepackage{newtxtext} 
\usepackage{enumitem}
\usepackage{amssymb}
\usepackage[table]{xcolor}
\usepackage{vwcol}
\usepackage{tikz}
\usetikzlibrary{arrows.meta}
\usetikzlibrary{calc}
\usepackage{mathtools}
\usepackage{nicematrix}
%For pictures / figures
\usepackage{color,graphicx,overpic}
\graphicspath{ {./images/} }
% for relations
\usepackage{cancel}
\usepackage{ mathrsfs }
\graphicspath{ {./images/} }
\setlist{nosep}


\pdfinfo{
  /Title (CG2111A-Midterm.pdf)
  /Creator (TeX)
  /Producer (pdfTeX 1.40.0)
  /Author (Seamus)
  /Subject (Example)
  /Keywords (pdflatex, latex,pdftex,tex)}

% Turn off header and footer
\pagestyle{empty}

\newenvironment{tightcenter}{%
  \setlength\topsep{0pt}
  \setlength\parskip{0pt}
  \begin{center}
}{%
  \end{center}
}

% redefine section commands to use less space
\makeatletter
\renewcommand{\section}{\@startsection{section}{1}{0mm}%
                                {-1ex plus -.5ex minus -.2ex}%
                                {0.5ex plus .2ex}%x
                                {\normalfont\large\bfseries}}
\renewcommand{\subsection}{\@startsection{subsection}{2}{0mm}%
                                {-1explus -.5ex minus -.2ex}%
                                {0.5ex plus .2ex}%
                                {\normalfont\normalsize\bfseries}}
\renewcommand{\subsubsection}{\@startsection{subsubsection}{3}{0mm}%
                                {-1ex plus -.5ex minus -.2ex}%
                                {1ex plus .2ex}%
                                {\normalfont\small\bfseries}}%
\renewcommand{\familydefault}{\sfdefault}
\renewcommand\rmdefault{\sfdefault}
% makes nested numbering (e.g. 1.1.1, 1.1.2, etc)
\renewcommand{\labelenumii}{\theenumii}
\renewcommand{\theenumii}{\theenumi.\arabic{enumii}.}
\renewcommand\labelitemii{•}
%  for logical not operator
\renewcommand{\lnot}{\mathord{\sim}}
\renewcommand{\bf}[1]{\textbf{#1}}
\newcommand{\abs}[1]{\vert #1 \vert}
\newcommand{\Mod}[1]{\ \mathrm{mod}\ #1}

\makeatother
\definecolor{myblue}{cmyk}{1,.72,0,.38}
\everymath\expandafter{\the\everymath \color{myblue}}
% Define BibTeX command
\def\BibTeX{{\rm B\kern-.05em{\sc i\kern-.025em b}\kern-.08em
    T\kern-.1667em\lower.7ex\hbox{E}\kern-.125emX}}
\let\iff\leftrightarrow
\let\Iff\Leftrightarrow
\let\then\rightarrow
\let\Then\Rightarrow

% Don't print section numbers
\setcounter{secnumdepth}{0}

\setlength{\parindent}{0pt}
\setlength{\parskip}{0pt plus 0.5ex}
%% this changes all items (enumerate and itemize)
\setlength{\leftmargini}{0.5cm}
\setlength{\leftmarginii}{0.5cm}
\setlist[itemize,1]{leftmargin=2mm,labelindent=1mm,labelsep=1mm}
\setlist[itemize,2]{leftmargin=4mm,labelindent=1mm,labelsep=1mm}

%My Environments
\newtheorem{example}[section]{Example}
% -----------------------------------------------------------------------

\begin{document}
\raggedright
\footnotesize
\begin{multicols}{4}


% multicol parameters
% These lengths are set only within the two main columns
\setlength{\columnseprule}{0.25pt}
\setlength{\premulticols}{1pt}
\setlength{\postmulticols}{1pt}
\setlength{\multicolsep}{1pt}
\setlength{\columnsep}{2pt}

\begin{center}
    \fbox{%
        \parbox{0.8\linewidth}{\centering \textcolor{black}{
            {\Large\textbf{CG2111A Final}}
            \\ \normalsize{AY24/25 sem 2}}
            \\ {\footnotesize \textcolor{myblue}{github.com/mendax1234}} 
        }%
    }
\end{center}

\section{LiDAR Programming}
\begin{enumerate}
    \item A communication session is \textbf{always} initialized by a \textbf{host system}, lie RPi or PC. RPiLiDAR \textbf{won't} send any data out automatically after powering up.
    \item Only \textbf{one response descriptor packet} will be sent out during a request/response session.
    \item The sample frequency for RPiLiDAR is 5.5 Hz. It takes 360 samples per revolution.
\end{enumerate}

\section{Communication Protocol}
\begin{enumerate}
    \item (\textbf{Magic number}): A ``magic number'' is a byte sequence that the receiver can use to detect \textbf{if the current block of data actually constitutes a data packet}, or it’s just simply a random sequence of bytes.
    \item \textbf{Checksum}: In CG2111A, the calculation of checksum is implemented using \texttt{XOR} operation.
    \begin{itemize}
        \item Checksums can \textbf{fail} to detect errors if there is an \textbf{even number of errors} in a column. (Think of the nature of \texttt{XOR})
        \item Checksum is stored using \textbf{one byte}.
    \end{itemize}
    \item \textbf{Send/Receive Data}:
    \begin{itemize}
        \item \textbf{Sender}: Construct the data to be sent from the lower memory address using endianness at the sender.
        \item \textbf{Receiver}: The bytes received are exactly the same as what are sent. Then use the endianness at the receiver to decide the data received.
    \end{itemize}
    \item In CG2111A, the data size issue only happens for \texttt{int} and \texttt{long}.
    \item \textbf{Padding}: In the exam, the padding bit should appear after the value bits, meaning lower memory address will store the valid value, higher memory address will store the padding bits. (If not state explicitly, clarify with the prof).
    \item \textbf{RPi} and \textbf{Arduino} both uses little-endian. RPi interprets \texttt{int} as 4 bytes, Arduino interprets it as 2 bytes.
\end{enumerate}

\section{Secure Networking}
\begin{enumerate}
    \item \textbf{TCP/UDP}: This is on \textbf{Transport layer}, it also adds the port to receive the packets
    \begin{itemize}
        \item \textbf{TCP}: more reliable because adding ``flow-control'' mechanism so that it a packet is corrupted, it needs to be resent!
        \item \textbf{UDP}: less reliable because corrupted packets will be dropped silently.
    \end{itemize}
\end{enumerate}

\section{Alex Programming}
\begin{enumerate}
    \item \textbf{Forward/Backward Distance}: \texttt{forwardDist = (leftForwardTicks / COUNTS\_PER\_REV * WHEEL\_CIRC)}, where \texttt{COUNTS\_PER\_REV} is the encoder counts when the wheel rotates one round. \texttt{WHEEL\_CIRC} is the length of one wheel round.
\end{enumerate}

\end{multicols}
\end{document}