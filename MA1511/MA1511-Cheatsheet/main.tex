\documentclass[10pt, landscape]{article}
\usepackage[scaled=0.92]{helvet}
\usepackage{calc}
\usepackage{multicol}
\usepackage{ifthen}
\usepackage[a4paper,margin=3mm,landscape]{geometry}
\usepackage{amsmath,amsthm,amsfonts,amssymb}
\usepackage{color,graphicx,overpic}
\usepackage{hyperref}
\usepackage{newtxtext} 
\usepackage{enumitem}
\usepackage{amssymb}
\usepackage[table]{xcolor}
\usepackage{vwcol}
\usepackage{tikz}
\usetikzlibrary{arrows.meta}
\usetikzlibrary{calc}
\usepackage{mathtools}
\usepackage{nicematrix}
%For pictures / figures
\usepackage{color,graphicx,overpic}
\graphicspath{ {./images/} }
% for relations
\usepackage{cancel}
\usepackage{ mathrsfs }
\graphicspath{ {./images/} }
\setlist{nosep}

\pdfinfo{
  /Title (MA1511.pdf)
  /Creator (TeX)
  /Producer (pdfTeX 1.40.0)
  /Author (Seamus)
  /Subject (Example)
  /Keywords (pdflatex, latex,pdftex,tex)}

% Turn off header and footer
\pagestyle{empty}

\newenvironment{tightcenter}{%
  \setlength\topsep{0pt}
  \setlength\parskip{0pt}
  \begin{center}
}{%
  \end{center}
}

% redefine section commands to use less space
\makeatletter
\renewcommand{\section}{\@startsection{section}{1}{0mm}%
                                {-1ex plus -.5ex minus -.2ex}%
                                {0.5ex plus .2ex}%x
                                {\normalfont\large\bfseries}}
\renewcommand{\subsection}{\@startsection{subsection}{2}{0mm}%
                                {-1explus -.5ex minus -.2ex}%
                                {0.5ex plus .2ex}%
                                {\normalfont\normalsize\bfseries}}
\renewcommand{\subsubsection}{\@startsection{subsubsection}{3}{0mm}%
                                {-1ex plus -.5ex minus -.2ex}%
                                {1ex plus .2ex}%
                                {\normalfont\small\bfseries}}%
\renewcommand{\familydefault}{\sfdefault}
\renewcommand\rmdefault{\sfdefault}
% makes nested numbering (e.g. 1.1.1, 1.1.2, etc)
\renewcommand{\labelenumii}{\theenumii}
\renewcommand{\theenumii}{\theenumi.\arabic{enumii}.}
\renewcommand\labelitemii{•}
%  for logical not operator
\renewcommand{\lnot}{\mathord{\sim}}
\renewcommand{\bf}[1]{\textbf{#1}}
\newcommand{\abs}[1]{\vert #1 \vert}
\newcommand{\Mod}[1]{\ \mathrm{mod}\ #1}

\makeatother
\definecolor{myblue}{cmyk}{1,.72,0,.38}
\everymath\expandafter{\the\everymath \color{myblue}}
% Define BibTeX command
\def\BibTeX{{\rm B\kern-.05em{\sc i\kern-.025em b}\kern-.08em
    T\kern-.1667em\lower.7ex\hbox{E}\kern-.125emX}}
\let\iff\leftrightarrow
\let\Iff\Leftrightarrow
\let\then\rightarrow
\let\Then\Rightarrow

% Don't print section numbers
\setcounter{secnumdepth}{0}

\setlength{\parindent}{0pt}
\setlength{\parskip}{0pt plus 0.5ex}
%% this changes all items (enumerate and itemize)
\setlength{\leftmargini}{0.5cm}
\setlength{\leftmarginii}{0.5cm}
\setlist[itemize,1]{leftmargin=2mm,labelindent=1mm,labelsep=1mm}
\setlist[itemize,2]{leftmargin=4mm,labelindent=1mm,labelsep=1mm}

%My Environments
\newtheorem{example}[section]{Example}
% -----------------------------------------------------------------------

\begin{document}
\raggedright
\footnotesize
\begin{multicols}{4}


% multicol parameters
% These lengths are set only within the two main columns
\setlength{\columnseprule}{0.25pt}
\setlength{\premulticols}{1pt}
\setlength{\postmulticols}{1pt}
\setlength{\multicolsep}{1pt}
\setlength{\columnsep}{2pt}

\begin{center}
    \fbox{%
        \parbox{0.8\linewidth}{\centering \textcolor{black}{
            {\Large\textbf{MA1511}}
            \\ \normalsize{AY24/25 sem 1}}
            \\ {\footnotesize \textcolor{myblue}{github.com/mendax1234}} 
        }%
    }
\end{center}

\section{01. Partial Derivatives}
\subsection{Level Curves}
\begin{enumerate}
    \item The projection of the contour curve onto $xy$-plane is a \textbf{level curve} of $f$. It consists of the \textbf{set} of points $(x,y)$ for which $f(x,y)$ has a constant value. \\
    For example, to find a typical level curve of $f(x,y)=yx^2$, just let $yx^2=k$, solve for $y$, we will get $y=k/x^2$
\end{enumerate}

\subsection{First Order Partial Derivatives}
\begin{enumerate}
    \item \textbf{(Rule of Differentiation)} $(\frac{f}{g})=\frac{f'g-g'f}{g^2}$ (quotient rule)
\end{enumerate}

\subsection{Higher Order Partial Derivatives}
\begin{enumerate}
    \item Given a function $f(x,y)$, $f_x(x,y)$ and $f_y(x,y)$ are both functions of $x$ and $y$. Thus differentiating these functions produces the so-called \textbf{second order partial derivatives} of $f$. For example, $f_{xy}$, $(f_x)_y$ or $\frac{\partial^2f}{\partial y \partial x}$ denotes $\frac{\partial}{\partial y}(\frac{\partial f}{\partial x})$.
    \item (\textbf{The Mixed Derivative Theorem}) For all functions covered in MA1511, we have $f_{xy} = f_{yx}$
\end{enumerate}

\subsection{Normal Lines and Tangent Planes}
\begin{enumerate}
    \item \textbf{(Tangent Plane)} A vector equation of the tangent plane at $P$ is $r\cdot(f_x(a,b), f_y(a,b), -1)=(a,b,f(a,b))\cdot(f_x(a,b), f_y(a,b), -1)$
    where $r=(x,y,z)$.
    Equivalently, a Cartesian equation of the plane is $z=f_x(a,b)\cdot (x-a) + f_y(a,b)\cdot (y-b) + f(a,b)$. It is achieved by doing the dot product on the vectors.
    \item \textbf{(Normal Lines)} The equation of the normal line at $P$ (which is the line passing through $P$ and perpendicular to the tangent plane) is $r=(a,b,f(a,b))+(f_x(a,b), f_y(a,b), -1)t, t\in R$. Notice that for line in $R^3$ or above, there is usually \textbf{no} single Cartesian equation for a line.
\end{enumerate}

\subsection{Chain Rule}
\begin{enumerate}
    \item \textbf{(Chain Rule with one independent variable)} If $z$ is a function of $x$ and $y$, and both $x$ and $y$ are functions of an \textbf{independent variable} $t$, then $\frac{dz}{dt} = \frac{\partial z}{\partial x} \times \frac{dx}{dt} + \frac{\partial z}{\partial y} \times \frac{dy}{dt}$
    \item \textbf{(Chain rule with two independent variables)} If $z$ is a function of $x$ and $y$, and both $x$ and $y$ are functions of two independent variables $s, t$ then $\frac{\partial z}{\partial s} = \frac{\partial z}{\partial x} \times \frac{\partial x}{\partial s} + \frac{\partial z}{\partial y} \times \frac{\partial y}{\partial s}$ and $\frac{\partial z}{\partial t} = \frac{\partial z}{\partial x} \times \frac{\partial x}{\partial t} + \frac{\partial z}{\partial y} \times \frac{\partial y}{\partial t}$
    \item \textbf{(Implicit Differentiation)}
    \begin{enumerate}
        \item (\textbf{Single Independent Variable}) $y=f(x)$ where $F(x,f(x))=0$, for all $x$ in the domain of $f$. We have $\frac{dy}{dx} =-\frac{F_x}{F_y}$
        \item (\textbf{Two Independent Variables}) If $\partial F / \partial z \neq 0$, we have $\frac{\partial z}{\partial x} = -\frac{F_x}{F_z}$ and $\frac{\partial z}{\partial y} = -\frac{F_y}{F_z}$
    \end{enumerate}
\end{enumerate}

\subsection{Directional Derivatives}
\begin{enumerate}
    \item (\textbf{Compute Directional Derivatives}) For any unit vector $u=u_1i+u_2j$, $D_uf(a,b) = u_1f_x(a,b) + u_2f_y(a,b)=(f_x(a,b), f_y(a,b))\cdot(u_1,u_2)$
    \item (\textbf{Significance of the Gradient Vector}) The gradient vector is defined as $\nabla f = \frac{\partial f}{\partial x} i + \frac{\partial f}{\partial y}j$
    \begin{enumerate}
        \item $\nabla f(x)$ points in the direction of maximum rate of increase of $f$ at $x$, and the maximum rate of change of $x$ is $|\nabla f(x)|$
        \item $\nabla f(x)$ is perpendicular to the \textbf{level curve or level surface} of $f$ through $x$.
    \end{enumerate}
\end{enumerate}

\subsection{Local Extrema}
\begin{enumerate}
    \item If $f$ has a \textbf{local maximum/minimum} at an \textit{interior point} $(a,b)$ of its domain, then $f_x(a,b) = 0 ~~\textbf{and}~~ f_y(a,b) = 0$ And $(a,b)$ is called the \textbf{critical point} of $f$.
    \item Saddle point is a \textbf{critical point}, but it is \textbf{neither a local maximum nor a local minimum}.
    \item To find the \textbf{absolute maximum/minimum}, we should find the \textbf{critical points} and \textbf{boundary points} and then compare the corresponding value.
    \item (\textbf{Derivative Test for Nature of Critical Points}) Let $(a,b)$ be a critical point of $f$. Let $D=f_{xx}(a,b)f_{yy}{a,b} - (f_{xy}(a,b))^2$.
    \begin{itemize}
        \item $f$ has a \textbf{local maximum} at $(a,b)$ if $D > 0$ and $f_{xx}(a,b) < 0$
        \item $f$ has a \textbf{local minimum} at $(a,b)$ if $D > 0$ and $f_{xx}(a,b) > 0$
        \item $f$ has neither a local maximum nor a local minimum point at $(a,b)$ if $D < 0$ (In this case, $(a,b)$ is known as a \textbf{saddle point})
    \end{itemize}
    The above test is \textbf{inconclusive} if $D = 0$
\end{enumerate}

\subsection{Lagrange Multipliers}
The maximum/minimum value of $f(x_1, x_2, \cdots, x_n)$ subject to the constraint $g(x_1, x_2, \cdots, x_n) = 0$ occurs at a point $(x_1, x_2, \cdots, x_n)$ that satisfies the following $(n+1)$ equations $f_{x_i}=\lambda g_{x_i}, i = 1,2,3,\cdots, n$ and $g(x_1, x_2, \cdots, x_n)=0$ for some constant $\lambda$, called \textbf{the Lagrange multiplier}. The solving method is
\begin{enumerate}
    \item Solve for the variables in $\lambda$
    \item Substitute the variables in $\lambda$ into the constraints
    \item Solve for $\lambda$ and substitute it back to get the variables' value
\end{enumerate}

\section{02. Multiple Integrals}
\subsection{Double Integrals over Rectangular Domain}
\begin{enumerate}
    \item (\textbf{Fubini's Theorem}) Let $f$ be a continuous function on the \textbf{rectangular domain} $\{(x,y):a\leq x \leq b, c \leq y \leq d\}$. Then $\int_a^b\int_c^df(x,y)dydx=\int_c^d\int_a^bf(x,y)dxdy$. When $f(x,y)=1$, the double integral $\int_D\int f(x,y)dA=\int_D\int 1 dA=\text{area of the region }D$
    \item (\textbf{A special case}) Let $f$ be defined on the \textbf{rectangular domain} $\{(x,y):a\leq x \leq b, c \leq y \leq d\}$. If $f(x,y)=a(x)b(y)$ for some continuous functions $a(x)$ and $b(y)$, then $\int_a^b\int_c^df(x,y)dydx = (\int_a^ba(x)dx)(\int_c^db(y)dy)$
\end{enumerate}

\subsection{Double Integrals over General Domains}
\begin{enumerate}
    \item (\textbf{Type I Domain}) Denoted as $\{(x,y): a \leq x \leq b, g(x) \leq y \leq h(x)\}$. The double integral is given by $\iint_Df(x,y)dA=\int_a^b\int_{g(x)}^{h(x)}f(x,y)dydx$.
    \item (\textbf{Type II Domain}) Denoted as $\{(x,y): c \leq y \leq d, g(y) \leq x \leq h(y)\}$ The double integral is given by $\iint_Df(x,y)dA=\int_c^d\int_{g(y)}^{h(y)}f(x,y)dxdy$. \\
    In conclusion, the variable with a constant boundary should be at \textbf{outer}. And \textbf{the order cannot be changed by just swapping the variables!}
    \item (\textbf{Steps to calculate double integrals})
    \begin{enumerate}
        \item Draw a diagram.
        \item For a Type I region, draw a \textbf{vertical line} from the lower boundary to the upper boundary
        \item For a Type II region, draw a \textbf{horizontal line} from the left boundary to the right boundary
    \end{enumerate}
    \item (\textbf{Change the order of Integration}) This means changing from Type I region to Type II region or from Type II region to Type I region.
\end{enumerate}

\subsection{Double Integrals in Polar Coordinates}
\begin{enumerate}
    \item (\textbf{The angle range in polar coordinates}) The point $P(x,y)$ in Cartesian coordinates can be represented by the ordered pair $(r, \theta)$ where $\theta = \alpha \text{ if } y \geq 0  \text{ or } -\alpha \text{ if } y < 0, \alpha \geq 0$
    \item (\textbf{Calculation}) If $f$ is continuous on a polar rectangle $R$ given by $\{(r, \theta): a \leq r \leq b, \alpha \leq \theta \leq \beta\}$, where $0 \leq \beta - \alpha \leq 2 \pi$. Then, $\int_R\int f(x,y)dA = \int_\alpha^\beta\int_a^bf(rcos\theta, rsin\theta)rdrd\theta$. The steps are below:
    \begin{enumerate}
        \item Use the geometric meaning of Double Integrals, which is the \textit{Area} $\cdot$ \textit{height}.
        \item Area is found by intersecting two planes (thus the range of $r, \theta$ can be determined).
        \item Height is the founded by using \textbf{upper surface $-$ lower surface}.
        \item Substitute $x=rcos\theta$ and $y=rsin\theta$ into the \textit{Height} function.
    \end{enumerate}
    Sometimes $a, b$ can be replaced by functions of $\theta$.
\end{enumerate}




\section{03. Vector Valued Functions}
\subsection{Curves and Motion in Space}
\begin{enumerate}
    \item A particle moving in the three-dimensional space, whose position $P(x,y,z)$ at time $t$ is described by three \textbf{parametric equations} $x=f(t), y=g(t), z=h(t)$ with $r(t)$ defined as $
        r(t) = f(t)i+g(t)j+h(t)k$. We call $r(t)$ a \textbf{vector-valued function} in one variable. The functions in $f(t), g(t), h(t)$ are the \textbf{components} of $r(t)$.
    \item The differentiation of a vector function $r(t)$ is done by \textbf{componentwise differentiation}. \\
    \item The magnitude of the velocity vector is $|r'(t)|=\sqrt{(f'(t))^2+(g'(t))^2+(h'(t))^2}$.
    \item A curve $r(t)=f(t)i + g(t)j + h(t)k$ is said to be \textbf{smooth} if it has no sharp corners (cusps). In MA1511, we only deal with smooth curves
    \item A \textbf{vector equation} of the \textbf{tangent line} to a curve $r(t)=f(t)i + g(t)j + h(t)k$ at the point where $t=t_0$ is $r = (f(t_0), g(t_0), h(t_0) + s(f'(t_0), g'(t_0), h'(t_0)), s \in R$
    in which $s$ is a parameter and each value of $s$ corresponds to a specific point on the tangent line.
    \item The line segment joining two distinct points, $A(x_1, y_1, z_1)$ and $B(x_2, y_2, z_2)$ has parametric representations
    $
        r(t)=(1-t)(x_1,y_1,z_1)+t(x_2,y_2,z_2)$ where $0 \leq t \leq 1$
\end{enumerate}

\subsection{Integrals of Vector-valued Functions}
\begin{enumerate}
    \item We define the indefinite integral $\int r(t)dt$ in terms of its component functions $f, g ~\text{and}~h$ by $\int r(t)dt=(\int f(t)dt)i + (\int g(t)dt)j + (\int h(t)dt)k$
    Note that we \textbf{do not use the same} integration constant for the three integrals.
    \item An application of this is that we can integrate the velocity vector to get the position vector. Similarly, we can integrate the acceleration vector to obtain the velocity vector.
\end{enumerate}

\subsection{Arc Length}
\begin{enumerate}
    \item The Length, $L$ of a smooth curve defined by the vector function $r(t) = f(t)i + g(t)j + h(t)k, a \leq t \leq b$ and traced exactly once as $t$ increases from $t=a$ to $t=b$, is given by $L=\int_a^b\sqrt{(f'(t))^2+(g'(t))^2+(h'(t))^2}dt$
\end{enumerate}

\subsection{Line Integrals}
\begin{enumerate}
    \item If $f(x,y)$ is defined on a smooth curve $C: r(t)=x(t)i+y(t)j, a \leq t \leq b$, the line integral of $f$ along $C$, denoted by $\int_Cf(x,y)ds$, is $\int_a^bf(x(t), y(t))\sqrt{(x'(t))^2+(y'(t))^2}dt$. We can think $\sqrt{(x'(t))^2+(y'(t))^2}dt$ as integrating \textit{speed} over time, it will produce \textit{distance} $ds$ and then we integrate \textit{distance} over height, it will produce \textit{area}.
    \item When $f(x,y)=1$, the line integral gives the length of the curve $C$.
    \item The three-dimensional variant is similar.
    \item When solving problems, find the parametric representation of the Curve $C$ in $t$, then substitute $x,y$ in $t$ back into the function $f$, then apply the formula.
\end{enumerate}

\subsection{Parametric Surfaces}
\begin{enumerate}
    \item A parametric surface is defined as a vector function of two variables $u$ and $v$, $r(u,v)=x(u,v)i + y(u,v)+ z(u,v)k$. For each $(u,v)$ in $D$, $r(u,v)$ represents the position vector of a point in space. These points constitute a surface.
    \item For a parametric surface, the vectors of partial derivatives $r_u, r_v$ are defined by $r_u=x_ui+y_uj+z_uk$ and $r_v=x_vi+y_vj+z_vk$.
    \item Given a smooth surface $r(u,v)$ and a point $P$ where $(u,v) = (u_0, v_0)$, the \textbf{normal vector to the tangent plane at $P$} is given by $(r_u \times r_v)(u_0, v_0)$, which means the vector $r_u \times r_v$ evaluated at $(u_0, v_0)$
\end{enumerate}

\section{04. Vector Fields}
\subsection{Vector Fields}
\begin{enumerate}
    \item A vector field in two dimensions is a two-dimensional vector whose component functions are functions of two variables. $F(x,y)=P(x,y)i + Q(x,y)j$
    Vector fields are vectors that depend on their \textbf{initial points}.
    \item In this chapter, a vector field, $F(x,y)$ (respectively $F(x,y,z)$) represents a variable \textbf{force} that depends on the position of the point $(x,y)$ (respectively $(x,y,z)$) at which it acts.
    \item The \textbf{gradient field (or gradient vector)}, denoted by $\nabla f$, is defined by $\nabla f=\frac{\partial f}{\partial x}i + \frac{\partial f}{\partial y}j$
\end{enumerate}

\subsection{Line Integrals of Vector Fields}
\begin{enumerate}
    \item (\textbf{The general method to calculate Line Integrals of Vector Fields}) The total work done by a vector field and $F(x,y)=P(x,y)i+Q(x,y)j$ in moving a particle along the curve $C: r(t)=x(t)i+y(t)j, a\leq r \leq b$, from the point $t=a$ (initial point) to the point $t=b$ (terminal point), denoted by $\int_CF\cdot dr$ is $\int_a^bF(r(t))\cdot r'(t)dt = \int_a^bP(x(t)), y(t))\cdot x'(t)+Q(x(t), y(t))\cdot y'(t)dt$ and $\int_CF\cdot dr$ is known as the \textbf{line integral} of the vector field $F$ along $C$
    \item (\textbf{The Steps for calculation}) Get the parametric equations of $r(t)$, then substitute the unknown variables with $\cdots t$.
\end{enumerate}

\subsection{Conservative Fields}
\begin{enumerate}
    \item A vector field $F$ is said to be \textbf{conservative} if there is a scalar function $f$ such that $F=\nabla f$. The scalar function $f$ is called a \textbf{potential function} of $F$. Obviously, if $f$ is potential function for $F$, then so is $(f+c)$ for any constant $c$.
    \item To find potential functions of a 2-D conservative field, let $f_x=P(x,y) ~\text{and}~ f_y=Q(x,y)$, then do partial integration, compare the results and add the \textbf{missing terms}. (3-D is similar)
\end{enumerate}

\subsection{Line Integrals in Conservative Fields}
\begin{enumerate}
    \item Given a \textbf{conservative field} $F=\nabla f$ for some differentiable scalar function $f$ and a smooth curve $C: r(t), a\leq t\leq b$ joining the point $r(a)$ to the point $r(b)$, the work done by $F$ in moving a particle along $C$ (or the line integral of $F$ along $C$) from $t=a$ to $t=b$ is $f(r(b))-f(r(a))$. \textbf{This means if $F$ is conservative, we have an easier way to calculate its line integral along a curve $C$}.
    \item Under the situation in the point 1, if the curve $C$ is closed, we have the work done/line integral to be $0$, which can be denoted as $\oint_CF\cdot dr=0$
    \item \textbf{(Test for conservative fields)}
    \begin{enumerate}
        \item $F(x,y,z)=P(x,y,z)i+Q(x,y,z)j+R(x,y,z)k$ is conservative \textbf{if and only if} $P_y=Q_x, Q_z=R_y\text{ and }R_x=P_z$
        \item $F(x,y)=P(x,y)i+Q(x,y)j$ is conservative \textbf{if and only if} $P_y=Q_x$
    \end{enumerate}
\end{enumerate}

\subsection{Green's Theorem}
\begin{enumerate}
    \item Let $F(x,y)=P(x,y)i+Q(x,y)j$. If $D$ is a region enclosed by a simple, closed and positively oriented curve $C$, then $\oint_CF\cdot dr=(\oint Pdx+Qdy)=\int\int_D(\frac{\partial Q}{\partial x}-\frac{\partial P}{\partial y})dA$. If the curve is negative oriented, then add a minus sign. And notice that the vector field \textbf{doesn't need to be conservative}. (\textbf{Green Theorem provides another way that uses double integral to calculate line integral of a vector field along a curve})
    \item The steps to use Green's Theorem is to find $P, Q$ first, form the double integrals. Then find the type of the region. Do the calculation on double integrals.
\end{enumerate}

\subsection{Curl and Divergence}
\begin{enumerate}
    \item Let $F$ be a vector field in three-dimensions. The curl of $F$, denoted by curl $F$, is a \textbf{vector} defined by $(R_y-Q_z)i+(P_z-R_x)j+(Q_x-P_y)k$
    \item The \textbf{divergence} of $F$, denoted by div $F$, is a \textbf{scalar} defined by $\text{div }F=P_x+Q_y+R_z$
\end{enumerate}

\section{05. Infinite Series}
\subsection{Sequences}
\begin{enumerate}
    \item For a sequence \{$a_n$\}, if $a_n\to 0 \text{ or a finite real number}$, then we say the sequence \{$a_n$\} is \textbf{convergent}. Otherwise, we say the sequence is \textbf{divergent} and it may diverge to \textbf{infinity} or diverge to nothing.
    \item When encountering a limit of the form $\frac{\infty}{\infty}$, divide the numerator and denominator by the highest power of $n$ that appears in the expression $\frac{P(n)}{Q(n)}$.
    \item \textbf{Some standard results on limits of sequences}
    \begin{enumerate}
        \item $lim_{n\to \infty} a^{\frac{1}{n}}=1$ fro any non-zero $a$
        \item $lim_{n\to \infty} n^{\frac{1}{n}}=1$
        \item $lim_{n\to \infty} r^n=0$ for $-1 \leq r \leq 1$
        \item $lim_{n\to \infty}(1+\frac{a}{n})^n=e^a$ for any $a \in R$
    \end{enumerate}
    \item (\textbf{Limit Laws}) Let \{$a_n$\} and \{$b_n$\} be two convergent sequences with $lim_{n\to \infty}a_n=A$ and $lim_{n\to \infty}b_n=B$
    \begin{enumerate}
        \item $lim_{n\to \infty}ca_n=cA$ and $lim_{n\to \infty}(c+a_n)=c+A$ for any real number $c$
        \item $lim_{n\to \infty}(a_n+b_n)=A+B$ and $lim_{n\to \infty}(a_n-b_n)=A-B$
        \item $lim_{n\to \infty}a_nb_n=AB$
        \item $lim_{n\to \infty}\frac{a_n}{b_n}=\frac{A}{B}$ provided $B\neq 0$ and $b_n \neq 0$ for all $n$
        \item $lim_{n\to \infty}f(a_n)=f(A)$ if $f$ is a function and the limit on the left exists
    \end{enumerate}
\end{enumerate}

\subsection{Infinite Series}
\begin{enumerate}
    \item Given a sequence \{$a_n$\}, and its $n^{th}$ partial sum $S_n$ is the sum of  its first $n$ terms. Since \{$S_n$\} is itself a sequence, we can consider the limit of \{$S_n$\} as $n$ tends to infinity, which is $\sum_{k=1}^{\infty}a_k$, this is called an \textbf{infinite series} with constant terms.
    \item The geometric series $\sum_{k=1}^{\infty}ar^{k-1}$ converges \textbf{if and only if} $-1 < r < 1$. Furthermore, for $-1 < r < 1$, $\sum_{k=1}^{\infty}ar^{k-1}=\frac{a}{1-r}$
    \item A tip to find the geometric series is to form $(\cdots x)^k$ and $(\cdots x)$ will be your $r$ in the formula above.
    \item Suppose $\sum a_k$ and $\sum b_k$ are two convergent series. Then, for any constants $\alpha, \beta, \sum (\alpha a_k+\beta b_k)$ converges and $\sum (\alpha a_k+\beta b_k) = \alpha \sum a_k + \beta \sum b_k$
    \item Let $m$ be a positive integer. Then, $\sum_{k=1}^{\infty}a_k$ is convergent if and only if $\sum_{k=m}^{\infty}a_k$ is convergent.
\end{enumerate}

\subsection{Two Convergence Tests for Infinite Series}
\begin{enumerate}
    \item (\textbf{$n^{th}$ Term Test, also known as Divergence Test}) If \{$a_n$\} does \textbf{not} converge to $0$, then the infinite series $\sum_{k=1}^{\infty}a_k$ is \textbf{divergent}. The converse of the test is \textbf{not true}.
    \item (\textbf{p-series Test}) The series $\sum_{k=1}^{\infty}\frac{1}{k^p}$ (known as p-series) is \textbf{convergent} if $p > 1$ and is \textbf{divergent} if $p \leq 1$
\end{enumerate}

\subsection{Power Series}
\begin{enumerate}
    \item In power series, our terms contain some variable, let's say $x$. The general form of power series is $\sum_{k=0}^{\infty}c_k(x-a)^k$, where $a$ is the \textbf{center} of the series. $c_1, c_2, \cdots$ are the \textbf{coefficients} of the series.
    \item Any given power series behaves in the following three ways
    \begin{enumerate}
        \item it diverges for all values of $x$ (other than $x=a$), its \textbf{radius of convergence} $R$ is $0$
        \item it converges for all values of $x$, its \textbf{radius of convergence} $R$ is $\infty$
        \item it converges if $|x-a|<R$ and diverges when $|x-a|>R$ for some positive real number $R$, called \textbf{radius of convergence} of the power series (\textbf{$a$ doesn't need to be 1})
    \end{enumerate}
    \item (\textbf{Ratio/Root Test for Power Series}) Methods to find the radius of convergence. Given a power series $\sum_{k=0}^{\infty}c_k(x-a)^k$, let $a_k=c_k(x-a)^k$
    \begin{enumerate}
        \item (\textbf{Ratio Test}) Let $L=lim_{k\to \infty}|\frac{a_{k+1}}{a_k}|$
        \item If $L < 1$(including $L=0$), the power series converges.
        \item If $L > 1$(including $L=\infty$, the power series diverges.
        \item (\textbf{Root Test}) Let $L=lim_{k\to \infty}|a_k^{1/k}|$
        \item If $L < 1$(including $L=0$), the power series converges.
        \item If $L > 1$(including $L=\infty$, the power series diverges.
    \end{enumerate}
\end{enumerate}

\subsection{Taylor Series}
\begin{enumerate}
    \item The \textbf{Taylor Series} for $f$ centered at $a$ is given by the infinite power series $\sum_{k=0}^{\infty}\frac{f^{(k)}(a)}{k!}(x-a)^k$. This is a power series whose coefficients are given by $\frac{f^{(k)}(a)}{k!}$. (When encountering terms like this , think about \textbf{Taylor Series} and \textbf{Maclaurin series}).
    \item The Taylor series of $f$ centered at $a=0$, $\sum_{k=1}^{\infty}\frac{f^{(k)}(0)}{k!}x^k$, is called the \textbf{Maclaurin series} of $f$.
    \item Some examples of functions with their associated Maclaurin series ($a=0$)
    \begin{enumerate}
        \item $\frac{1}{1-x}=\sum_{k=0}^{\infty}x^k=1+x+x^2+x^3+\cdots$ for $-1 < x <1$
        \item $\frac{1}{1+x}=\sum_{k=0}^{\infty}x^k=1-x+x^2+x^3+\cdots+(-x)^n+\cdots$ for $-1 < x <1$
        \item $e^x=\sum_{k=0}^{\infty}\frac{1}{k!}x^k=1+x+\frac{x^2}{2!}+\frac{x^3}{3!}+\cdots$ for all $x$
        \item $sinx=\sum_{k=0}^{\infty}\frac{(-1)^k}{(2k+1)!}x^{2k+1}=x-\frac{x^3}{3!}+\frac{x^5}{5!}+\cdots$ for all $x$
        \item $cosx=\sum_{k=0}^{\infty}\frac{(-1)^k}{(2k)!}x^{2k}=1-\frac{x^2}{2!}+\frac{x^4}{4!}+\cdots$ for all $x$
        \item $tan^{-1}x=x-\frac{x^3}{3}+\frac{x^5}{5}-\frac{x^7}{7}+\cdots$ for $-1\leq x\leq 1$
        \item $ln(1+x)=\sum_{k=0}^{\infty}\frac{(-1)^{k+1}}{k}x^k=x-\frac{x^2}{2}+\frac{x^3}{3}+\cdots$ for $-1 < x \leq 1$
        \item $(1+x)^p=\sum_{k=0}^{\infty}
        \begin{pmatrix}
            p \\
            k
        \end{pmatrix}x^k=1+px+\frac{p(p-1)}{2!}x^2+\frac{p(p-1)(p-2)}{3!}x^3+\cdots$ for $-1<x<1$, where $\begin{pmatrix}
            p \\
            0
        \end{pmatrix}=1, \begin{pmatrix}
            p \\
            k
        \end{pmatrix}=\frac{p(p-1)(p-2)\cdots(p-k+1)}{k!}, k=1,2,3\cdots$
    \end{enumerate}
    \item (\textbf{Taylor Polynomial}) Given a function $f$, its $n^{th}$ order Taylor polynomial centered at $a$, denoted by $P_n$, is the sum of the first $(n+1)$ terms of its Taylor Series $\sum_{k=0}^{\infty}\frac{f^{(k)}(a)}{k!}(x-a)^k$. That is, $P_n(x)=\sum_{k=0}^{\infty}\frac{f^{(k)}(a)}{k!}(x-a)^k$
\end{enumerate}

\end{multicols}

% Dividing Line
\hrulefill \\

\begin{multicols}{3}
    \begin{enumerate}
        \item When doing higher order differentiation questions, if it is hard to differentiate at first, try using \textbf{Mixed Derivatives Theorem}.
        \item For \textbf{Double Integrals}, if it is hard to integrate, consider changing the order of integration. But pay attention to the region Type.
        \item When changing region type, notice that you may need to change the function expression also. For example, $y=\cdots$ may become $x=\cdots$.
        \item When trying to get the formula for \textbf{polar curves}, if it is not obvious, try to time $r$ to see if can substitute with $x=rcos\theta, y=rsin\theta, r^2=x^2+y^2$
        \item For problems where two or more space curves are involved, different symbols should be used when finding the \textbf{intersection} of the curves. And only when all these different variables share the same value can we say the two or more curves intersect. Otherwise, they don't collide.
        \item The vector cross product's sign is $+, -, +$
        \item Given a gradient field/gradient vector with an unknown constant, try using \textbf{Mixed Derivative Theorem} to solve for the unknown constant.
        \item In vector-field, it is important if you denote the $P, Q$ parts at first, then it will be easier to substitute the formulas in.
    \end{enumerate}
\end{multicols}

\end{document}
