\documentclass[math,code]{amznotes}
\setcounter{tocdepth}{2}  % Only show subsections and sections in the ToC
\usepackage[utf8]{inputenc}
\usepackage{amsmath}
\usepackage{amsfonts}
\usepackage{graphicx}
\usepackage{tikz}
\usepackage{etoolbox}
\usepackage{tabularx}
\usepackage{float} % Needed for [H] placement specifier
\usepackage{wrapfig} % Needed for wrapping figures
\usepackage{diagbox} % For diagonal split in table
\usepackage{booktabs} % For better table formatting
\usepackage[normalem]{ulem} % For table usage
\useunder{\uline}{\ul}{} % For table usage
\usepackage{multirow}

\graphicspath{ {./images/} }
\geometry{
    a4paper,
    headheight = 1.5cm
}

\patchcmd{\chapter}{\thispagestyle{plain}}{\thispagestyle{fancy}}{}{}

\theoremstyle{remark}
\newtheorem*{claim}{Claim}
\newtheorem*{remark}{Remark}
\newtheorem{case}{Case}

\newcommand{\entry}[5]{\markboth{#1}{#1}\textbf{#1}\ {(#2)}\ \textit{#3}\ $\bullet$\ \textbf{\textit{#4}} \begin{small}#5\end{small}}  % Defines the command to print each word on the page, \markboth{}{} prints the first word on the page in the top left header and the last word in the top right

%----------------------------------------------------------------------------------------

\begin{document}
\fancyhead[L]{
    French 4
}
\fancyhead[R]{
    Lecture Notes
}
\tableofcontents
\chapter{Unité 5}
\section{La place des adjectifs}
L'adjectif donne une information ou une précision sur quelque chose ou quelqu'un.
\begin{enumerate}
    \item En général, l'adjectif se place \textcolor{orange}{\textbf{après le nom}}.
    \item Mais certains adjectifs se placent \textcolor{orange}{\textbf{avant le nom}}.
        \begin{table}[h]
            \centering
            \begin{tabularx}{\textwidth}{|X|X|X|}
            \hline
            \textbf{Les adjectifs courts et fréquents} & petit, grand, gros, beau, joli, bon, mauvais, vieux, autre, nouveau & C'est un \textbf{beau} voyage. \\
            \hline
            \textbf{ Les adjectifs qui indiquent un classement} & premier, deuxième, troisième, dernier & Il habite au \textbf{deuxième} étage. \\
            \hline
            \end{tabularx}
        \end{table}
        \begin{notebox}
            \textbf{Attention}:
                \begin{enumerate}
                    \item Such kind of rules are summarized into a rule called, ``BAGS'' (\textbf{B}eauty, \textbf{A}ge, \textbf{G}ood, \textbf{S}ize). Les adjectifs comme, \textit{petit, grand, gros, beau, joli, bon, mauvais, vieux, autre, nouveau, ...}
                    \item Devant un nom masculin qui commence par une voyelle ou un \textit{h} meut:
                    \begin{center}
                        beau $\rightarrow$ bel, vieux $\rightarrow$ vieil, nouveau $\rightarrow$ nouvel
                    \end{center}
                    e.g., C'est un \textbf{bel} \underline{h}omme!
                    \item Pour l’année, le mois ou la semaine, \textit{dernier} est après le nom. \\
                    e.g., La semaine \textbf{dernière}, je suis allé en France.
                \end{enumerate}
        \end{notebox}
    \item Certains \textcolor{orange}{\textbf{changent de sens}} s'ils sont places avant ou après le nom.    
    \begin{table}[h]
    \centering
    \begin{tabular}{|l|l|l|}
    \hline
    \multicolumn{1}{|c|}{ancien}  & \begin{tabular}[c]{@{}l@{}}qui était mais n'est plus\\ qui est vieux\end{tabular}        & \begin{tabular}[c]{@{}l@{}}mon ancienne maison\\ une maison ancienne\end{tabular}     \\ \hline
    \multicolumn{1}{|c|}{curieux} & \begin{tabular}[c]{@{}l@{}}étrange, bizarre\\ qui a de l’intérêt pour\end{tabular}       & \begin{tabular}[c]{@{}l@{}}une curieuse personne\\ une personne curieuse\end{tabular} \\ \hline
    grand                         & \begin{tabular}[c]{@{}l@{}}célébré\\ de grande taille\end{tabular}                       & \begin{tabular}[c]{@{}l@{}}un grand homme\\ un homme grand\end{tabular}               \\ \hline
    nouveau                       & \begin{tabular}[c]{@{}l@{}}différent de ceux que l'on a\\ imprimé récemment\end{tabular} & \begin{tabular}[c]{@{}l@{}}un nouveau livre\\ un livre nouveau\end{tabular}           \\ \hline
    seul                          & \begin{tabular}[c]{@{}l@{}}unique\\ solitaire\end{tabular}                               & \begin{tabular}[c]{@{}l@{}}un seul homme\\ un homme seul\end{tabular}                 \\ \hline
    \end{tabular}
    \end{table}
    \begin{notebox}
        \textbf{Attention:}
        \begin{enumerate}
            \item Quand un adjectif pluriel est devant un nom, \textit{des} $\rightarrow$ de. \\
            e.g., Il apprend \textbf{de} nouvelles choses.
            \item A l'oral, si on veux insister sur l'adjectif, on le place avant le nom.
        \end{enumerate}
    \end{notebox}
\end{enumerate}

\section{Les pronoms en et y}
Les pronoms \textit{en} et \textit{y} sont des pronoms \textbf{indirects}.
\begin{enumerate}
    \item \textcolor{orange}{\textbf{en}} remplace un nom qui exprime \textcolor{orange}{\textbf{une quantité}}. Si la quantité est précise, on peut l'indiquer. \\
    \underline{Des pas}\footnote{pas: footsteps}? J'\textbf{en} fais tous les jours. \\
    \underline{Des pas}? Mon application \textbf{en} compte 12 345.
    \item \textcolor{orange}{\textbf{en}} remplace une chose, une idée ou un animal avec \textcolor{orange}{\textbf{un verbe construit avec la préposition \textit{de}}}. \\
    e.g., Ils sont fiers \underline{de} cette victoire $\rightarrow$ Ils \textbf{en} sont fiers. (être fier de\footnote{be pround of} quelque chose)
    \item \textcolor{orange}{\textbf{y}} remplace \textcolor{orange}{\textbf{un lieu}} \\
    e.g., On allait \underline{au bureau} à pied. $\rightarrow$ On \textbf{y} allait à pied.
    \begin{notebox}
            Quand il s'agit d'un lieu d’où l'on vient, on utilise \textbf{en} car le verbe utilise \textit{de}. \\
            e.g., On revient \underline{de} Paris. $\rightarrow$ On \textbf{en} revient. 
    \end{notebox}
    \item \textcolor{orange}{\textbf{y}} remplace une chose, une idée ou un animal avec \textcolor{orange}{\textbf{un verbe construit avec la préposition \textit{à}}}. \\
    e.g., L’équipe a participe \underline{à} cet événement $\rightarrow$ Elle \textbf{y} a participe. (participer à quelque chose)
    \begin{notebox}
        La regle est la meme avec \textit{au} (= à + le). \\
        e.g., L'équipe a participé \underline{au} défi\footnote{défi: challenge}. $\rightarrow$ Elle \textbf{y} a participé.
    \end{notebox}
    \item Les pronoms \textcolor{orange}{\textbf{en}} et \textcolor{orange}{\textbf{y}} se placent généralement \textbf{avant le verbe}. \\
    e.g., J'\textbf{y} vais. J'\textbf{en} acheté. N'\textbf{en} achète pas!
    \begin{notebox}
        \textbf{Attention}
        \begin{enumerate}
            \item Sauf \footnote{sauf: except that} à l’impératif et c'est à la forme affirmative. e.g., Vas-\textbf{y}! Mais, à la forme négative, c'est normal, N’\textbf{y} va pas!
            \item On ajoute un \textit{-s} aux verbes en \textit{-er}, à la $2^\text{e}$ personne du singulier. e.g., Mange\textbf{s-en}! Achète\textbf{s-en} trois!
        \end{enumerate}
    \end{notebox}
    \item \textbf{Plusieurs exemples classiques}
    \begin{itemize}
        \item N'oublie pas de m'\textbf{en} envoyer \underline{une ou deux}.
        \item Tu te souviens de ce voyage? $\rightarrow$ Je ne m'\textbf{en} souviens pas.
        \item Je voudrais participer à ce défi. Et toi? $\rightarrow$ Oui, je voudrais \textbf{y} participer.
        \item Je rêve d'aller en Australie. Et toi? $\rightarrow$ Oui, j'\textbf{en} rêve./Non, je \textbf{n'en} rêve pas.
        \item Je vais manger des pommes. $\rightarrow$ Oui, je vais \textbf{en} manger./ Non, je ne vais pas \textbf{en} manger.
        \item Je me suis occupé de ça $\rightarrow$ Je m'\textbf{en} suis occupé.
    \end{itemize}
\end{enumerate}

\section{La négation}
La négation a deux éléments qui entourent le verbe conjugué \footnote{that surround the conjugated verb}.
\begin{center}
    \textcolor{orange}{\textbf{ne}} + verbe conjugué + \textcolor{orange}{\textbf{pas}}
\end{center}
\begin{enumerate}
    \item Pour exprimer une négation, on peut utiliser:
    \begin{table}[h]
        \centering
        \begin{tabular}{|l|l|}
            \hline
            \textbf{négations} & \textbf{exemples} \\ \hline
            ne...pas & Ça \textbf{ne} change \textbf{pas}. \\ \hline
            ne...plus ($\neq$ encore) & Ça \textbf{ne} change \textbf{plu}s. \\ \hline
            ne...jamais ($\neq$ toujours) & Ça \textbf{ne} change \textbf{jamais}. \\ \hline
            ne...personne ($\neq$ quelqu'un) & Ça \textbf{ne} change \textbf{personn}e. \\ \hline
            ne...rien ($\neq$ quelque chose) & Ça \textbf{ne} change \textbf{rien}. \\ \hline
            ne...ni...ni (pour deux éléments) & Ça \textbf{ne} change \textbf{ni} ma décision \textbf{ni} celle de ton père. \\ \hline
        \end{tabular}
    \end{table}
    \item Au \textcolor{orange}{\textbf{futur proche}}:
    \begin{center}
        sujet + \textcolor{orange}{\textbf{ne}} + aller + \textcolor{orange}{\textbf{pas}} + verbe à l'infinitif
    \end{center}
    e.g., Ça \textbf{ne} va \textbf{jamais} changer.
    \item Au \textbf{passé composé}
    \begin{center}
        sujet + \textcolor{orange}{\textbf{ne}} + avoir/être + \textcolor{orange}{\textbf{pas}} + participe passé
    \end{center}
    e.g., Ça \textbf{n'}a \textbf{jamais} changé.
    \begin{notebox}
        Sauf avec \textit{ne$\cdots$ personne} $\rightarrow$ sujet + \textbf{ne} + avoir/être + participe passé + \textbf{personne} \\
        e.g., Ça \textbf{n}'a changé \textbf{personne}.
    \end{notebox}
    \item A l'oral, on ne prononce pas toujours le \textit{ne}. \\
    e.g., Ça (ne) change pas!
    \begin{notebox}
        C'est pour \textit{pas} seulement.
    \end{notebox}
    \item \textbf{Plusieurs exemples classiques}
    \begin{itemize}
        \item Il \textbf{ne} faut \textbf{rien} manger de gras.
        \item Tu \textbf{ne} dois \textbf{jamais} rater\footnote{rater: miss} ton entraînement\footnote{entraînement: training}.
        \item Il \textbf{ne} faut \textbf{ni} manger trop \textbf{ni} se coucher\footnote{se coucher: go to bed} tard.
        \item Thomas Pesquet \textbf{ne} s'est \textbf{jamais} marié.
        \item Il \textbf{ne} va \textbf{plus} retourner dans la Station spatiale internationale.
        \item Dans mon equipe, tout le monde a un handicap. $\rightarrow$ $\cdots$, \textbf{personne n'}a un handicap.
        \item Avant, il y avait beaucoup d'articles. $\rightarrow$ Avant, il \textbf{n'}y avait \textbf{aucun}\footnote{aucun: none} article.
    \end{itemize}
\end{enumerate}

\section{Le préfixe pour le contraire}
On ajoute \textbf{dé- / dés- / in- / im- / ir- / il- / mal-} devant un adjectif pour former le contraire.
\begin{enumerate}
    \item Le choix dépend souvent de la lettre qui suit pour que ce soit plus facile à prononcer\footnote{The choice often depends on the letter that follows to make it easier to pronounce.}
    \begin{itemize}
        \item \textbf{im-} devant \textit{m} ou \textit{p} (e.g., impatient, impossible)
        \item \textbf{ir-} devant \textit{r} (e.g., irresponsable, irréel)
        \item \textbf{il-} devant \textit{l} (e.g., illogique, illisible)
        \item \textbf{in-} dans les autres cas (e.g., incapable, injuste)
        \item \textbf{dé- / dés-} pour certains \textbf{verbes/adjectifs} (e.g., désorganisé, démodé)
        \item \textbf{mal-} parfois pour donner une idée de ``\textit{mal fait}''\footnote{mal fait: badly done} (e.g., malhonnête, maladroit).
    \end{itemize}
    \item \textbf{Plusieurs exemples classiques}: \textbf{mal}honnête $\rightarrow$ honnête, \textbf{dés}agréable $\rightarrow$ agréable, \textbf{dés}ordonné $\rightarrow$ ordonné, \textbf{mal}adroit $\rightarrow$ adroit, \textbf{ir}responsable $\rightarrow$ responsable, \textbf{im}poli $\rightarrow$ poli, \textbf{im}prudent $\rightarrow$ prudent\footnote{prudent: careful, cautious}
\end{enumerate}

\chapter{Unité 6}
\section{Le superlatif}
Le superlatif exprime la supériorité ou l'infériorité.
\begin{table}[h]
    \begin{tabularx}{\textwidth}{|X|X|X|}
    \hline
                                      & \textbf{La supériorité} (+)                           & \textbf{L'infériorité} (-)           \\ \hline
    \multirow{3}{*}{\textbf{avec un adjectif}} & Cet exercice est \textbf{le plus} difficile.                   & Cet exercice est \textbf{le moins} difficile.                  \\ \cline{2-3} 
                                      & Hiba est \textbf{la plus} grande.                              & Xiao est \textbf{la moins} grande.                             \\ \cline{2-3} 
                                      & Ce sont \textbf{les plus} sérieux de la classe.                & Ce sont \textbf{les moins} sérieux de la classe.               \\ \hline
    \textbf{avec un adverbe}                   & C'est l'homme qui court \textbf{le plus} vite.                 & C'est l'homme qui court \textbf{le moins} vite.                \\ \hline
    \textbf{avec un nom}                       & Paris est la ville qui a \textbf{le plus} de touristes par an. & C'est le pays où il y a \textbf{le moins} de touristes par an. \\ \hline
    \textbf{avec un verbe}                     & C'est elle qui mange \textbf{le plus}.                         & C'est lui qui mange \textbf{le moins}.                         \\ \hline
    \end{tabularx}
\end{table}
\begin{notebox}
    \textbf{Attention}! 
    \begin{enumerate}
        \item Quand l'adjectif est après le nom, il faut répéter l'article défini avant l'adjectif. \\
        e.g., Hiba est \textbf{la} femme \textbf{la plus} grande de la classe.
        \item Plusieurs exemples classiques
        \begin{itemize}
            \item bon $\rightarrow$ le(s), la meilleur(e)(s). e.g., C'est \textbf{le meilleur} moyen d'apprendre!
            \item mauvais $\rightarrow$ le(s), la pire(s). e.g., Ce sont \textbf{les pires} choses à manger!
            \item bien $\rightarrow$ le(s), la mieux. e.g., C'est \textbf{le mieux} habillé.
        \end{itemize}
    \end{enumerate}
\end{notebox}
\begin{enumerate}
    \item Le superlatif peut avoir un complément introduit par \textcolor{orange}{\textbf{de}}. \\
    e.g., C'est le plus beau jour \textbf{de} ma vie!
    \item L'article (l\textit{e, la, les}) peut-être remplace par un \textbf{possessif}. \\
    e.g., C'est la meilleure amie de Chloé. $\rightarrow$ C'est \textbf{sa} meilleure amie.
    \item \textbf{Plusieurs exemples classiques}
    \begin{itemize}
        \item Ce sont les épices\footnote{épice: spice} \textbf{les plus savoureuses}\footnote{savoureuse: delicious}.
    \end{itemize}
\end{enumerate}

\section{Les pronoms interrogatifs}
Pour interroger\footnote{interroger: query} parmi\footnote{parmi: among} des éléments déjà cites, on utilise \textit{lequel}, \textit{laquelle}, \textit{lesquels}, \textit{lesquelles}.
\begin{enumerate}
    \item Le pronom s'accorde en \textcolor{orange}{\textbf{genre}} et en \textcolor{orange}{\textbf{nombre}} avec l’élément cite et la réponse souhaitée (une ou des réponses)
    \begin{table}[h]
        \centering
        \begin{tabular}{|l|l|l|}
        \hline
         & \textbf{masculin} & \textbf{féminin} \\ \hline
        \textbf{singulier} & lequel & laquelle \\ \hline
        \textbf{pluriel} & lesquels & lesquelles \\ \hline
        \end{tabular}
    \end{table}
    
    \underline{Ces machines} sont extraordinaires. (féminin, pluriel). Vous avez préféré \textbf{laquelle}? (= une seule réponse) \\
    Parmi \underline{ces appréciations.} (féminin, pluriel) \textbf{lesquelles} entendez-vous? (= plusieurs réponses)
\end{enumerate}

\section{Les adverbes}
L'adverbe donne une précision sur un verbe, un adjectif ou un autre adverbe. Il est invariable.
\begin{notebox}
    \begin{remark}
        L'adverbe n'ai pas besoin de s'accorder avec le \textcolor{orange}{\textbf{nombre}} et le \textcolor{orange}{\textbf{genre}}.
    \end{remark}
\end{notebox}
\begin{enumerate}
    \item En général, on ajoute \textcolor{orange}{\textbf{-ment}} au \textcolor{orange}{\textbf{féminin de l'adjectif}}. \\
    Écoute \textbf{attentivement}. (attentif $\rightarrow$ attentive $\rightarrow$ attentivement) \\
    Rédige\footnote{Rédiger: write} ce texte \textbf{immédiatement}. (immédiat $\rightarrow$ immédiate $\rightarrow$ immédiatement)
    \item On ajoute \textcolor{orange}{\textbf{-ment}} au \textcolor{orange}{\textbf{masculin}} de l'adjectif termine par une \textcolor{orange}{\textbf{voyelle}}. \\
    Il est \textbf{vraiment} incroyable! (var\underline{i} + -ment $\rightarrow$ vraiment) \\
    Il est \textbf{incroyablement} beau. (incroyabl\underline{e} + -ment $\rightarrow$ incroyablement
    \item Quand l'adjectif masculin se termine par \textcolor{orange}{\textbf{-ant}} ou \textcolor{orange}{\textbf{-ent}}, l'adverbe se termine par \textcolor{orange}{\textbf{-amment}} ou \textcolor{orange}{\textbf{-emment}}. \\
    Prends \textbf{suffisamment} le temps! (suffis\underline{ant} $\rightarrow$ suffisamment) \\
    As-tu lu \textbf{récemment} un livre? (réc\underline{ent} $\rightarrow$ récemment) 
    \begin{notebox}
        \textbf{Attention!} Il existe quelques cas particuliers:
        \begin{multicols}{2}
            précis $\rightarrow$ précisément
            
            profond $\rightarrow$ profondément

            énorme $\rightarrow$ énormément

            intense $\rightarrow$ intensément
        \end{multicols}
    \end{notebox}
    \item L'adverbe se place \textcolor{orange}{\textbf{avant un adjectif ou un autre adverbe}}. \\
    e.g., Il est \textbf{vraiment} drôle.
    \item L'adverbe se place \textbf{après un verbe} au présent, imparfait, futur simple, etc. mais \textcolor{orange}{\textbf{entre l'auxiliaire et le participe passé}} au passé composé. Le futur proche aussi. Mais l’adverbe se place après le verbe pronominal (e.g., se placer), \\
    Je travaille \textbf{énormément}. \\
    J'ai \textbf{énormément} travaille.
    \item \textbf{Plusieurs exemples classiques}
    \begin{itemize}
        \item Ma collègue va \textbf{sûrement} prendre des congés\footnote{congé: leave}.
        \item Cette pause gourmande me fait \textbf{extrêmement} plaisir.
        \item Elle est \underline{bruyante}. $\rightarrow$ Elle parle \textbf{bruyamment}.
    \end{itemize}
\end{enumerate}

\chapter{Unité 7}
\section{Passé composé et l'imparfait}
Dans ce section, nous allons discuter de la différence entre le passé composé et l'imparfait.
\begin{enumerate}
    \item \textcolor{orange}{\textbf{L'imparfait}} indique \textcolor{orange}{\textbf{la situation}} de départ\footnote{starting situation}. \\
    e.g., C'\textbf{était} en 2021. J'\textbf{étais} étudiant à l’université.
    \item \textcolor{orange}{\textbf{L'imparfait}} indique \textcolor{orange}{\textbf{une habitude}} dans le passé et qui ne se passe pas maintenant. \\
    Petit, j'\textbf{allais} souvent au marche avec ma grand-mère.\\
    Tous les dimanches, nous \textbf{mangions} un poulet rôti\footnote{rôti: roasted}.
    \item \textcolor{orange}{\textbf{L'imparfait}} présente \textcolor{orange}{\textbf{une émotion, une description}} \\
    e.g., Le public \textbf{était} ravi. Elle \textbf{avait} un magnifique sourire\footnote{sourire: smile}.
    \item \textcolor{orange}{\textbf{Le passé composé}} présente une \textcolor{orange}{\textbf{action passée}} à un moment précis. \\
    e.g., Elle \textbf{a présente} sa première collection à New York.
    \item \textcolor{orange}{\textbf{Le passé composé}} présente \textcolor{orange}{\textbf{une continuité}} avec le présent. \\
    e.g., Elle s'\textbf{est} toujours \textbf{inspirée} de son héritage.
    \item \textcolor{orange}{\textbf{Le passé composé}} indique \textcolor{orange}{\textbf{un changement}} dans la situation. \\
    e.g., Elle \textbf{a} finalement \textbf{changé} de voie\footnote{voie n.f.: way}.
    \item Dans un récit\footnote{récit: story}, on utilise le passé composé  et/ou l'imparfait selon les actions et les situations décrites. \\
    e.g., Il \textbf{était} à l’université quand il \textbf{a décidé} d’être styliste. 
\end{enumerate}
\begin{notebox}
    \textbf{Attention!}
    \begin{enumerate}
        \item Les mots comme \textcolor{orange}{\textbf{tout à coup, soudain, brusquement\footnote{suddenly}, un jour, a ce moment-là, alors, d'abord, puis, après, ensuite}}, etc, qui indiquent un changement dans le récit, s'utilisent avec \textbf{le passé composé}. \\
        e.g., Je \textbf{dormais} profondément quand, \underline{tout à coup}, un bruit violent m'\textbf{a réveille}.
        \item Pour \textbf{l'imparfait}, il y a quelque mots clés: \textcolor{orange}{\textbf{souvent, toujours, normalement, d'habitude, tous les jours, chaque jour, pendant que, quelque fois, rarement}}, etc.
        \item Pour le verbe pronominal, comme \textcolor{orange}{\textbf{se laver}}, son imparfait est comme
        \begin{center}
            Sujet + pronom réfléchi + être + participe passé
        \end{center}
        e.g., Elles se sont rencontrées.
    \end{enumerate}
\end{notebox}

\section{Le pronom indirect et direct}
En français, il y a deux types de pronoms compléments
\begin{enumerate}
    \item le pronom complément d’objet direct (\textit{COD})
    \item le pronom complément d’objet indirect (\textit{COI})
\end{enumerate}
\subsection{Le COD et le pronom direct}
Le COD répond à la question ``qui ?'' ou ``quoi ?'' après le verbe, sans préposition. Par exemple,
\begin{center}
    Je vois Marie. $\rightarrow$ Je vois \textbf{qui}? $\rightarrow$ Marie (Marie, c'est COD ici)
\end{center}
On remplace le COD (Marie) par un pronom direct :
\begin{center}
    Je \textbf{la} vois.
\end{center}
Est le règle est suivant
\begin{table}[h]
    \centering
    \begin{tabular}{|c|c|c|}
    \hline
    \textbf{masculin singulier} & \textbf{féminin singulier} & \textbf{pluriel} \\ \hline
    le                 & la                & les     \\ \hline
    \end{tabular}
\end{table}

Pour plus de exemples, nous avons suivant
\begin{itemize}
    \item Il mange \textbf{la pomme} $\rightarrow$ Il \textbf{la} mange.
    \item Nous aimons \textbf{les films} $\rightarrow$ Nous \textbf{les} aimons.
    \item Te vois le prof $\rightarrow$ Te \textbf{le} vois.
\end{itemize}
\subsection{Le COI et le pronom indirect}
Le COI répond à la question ``à qui ?'' ou ``à quoi ?'' après le verbe, et il a toujours une préposition (souvent ``à''). Par exemple,
\begin{center}
    Le professeur permet \textbf{aux étudiants} de manger en classe. \\$\rightarrow$ Le professeur permet \textbf{à qui} ? $\rightarrow$ \textbf{aux étudiants} (C'est COI ici)
\end{center}
On remplace aux étudiants par le pronom indirect ``leur'' :
\begin{center}
    Le professeur \textbf{leur} permet de manger en classe.
\end{center}
Est le règle est suivant
\begin{table}[h]
    \centering
    \begin{tabular}{|c|c|}
    \hline
    \textbf{singulier}   & \textbf{pluriel}       \\ \hline
    lui (à lui / à elle) & leur (à eux / à elles) \\ \hline
    \end{tabular}
\end{table}

Pour plus de exemples, nous avons suivant
\begin{itemize}
    \item Je parle \textbf{à Paul} $\rightarrow$ Je \textbf{lui} parle.
    \item Tu écris \textbf{à tes amis} $\rightarrow$ Tu \textbf{leur} écris.
    \item Elle téléphone \textbf{à sa mère} $\rightarrow$ Elle \textbf{lui} téléphone.
\end{itemize}

\section{La cause et la conséquence}
On utilise les expressions de la \textcolor{orange}{\textbf{cause}} pour expliquer une situation. \\
e.g., Je suis malade \textbf{parce que} \underline{j'ai attrapé froid hier.}\footnote{explication}

On utilise les expressions de la \textcolor{orange}{\textbf{conséquence}} pour indiquer un résultat. \\
J'ai attrape froid hier. \textbf{Donc}, \underline{je suis malade aujourd'hui.}\footnote{résultat}

\begin{enumerate}
    \item \textcolor{orange}{\textbf{La cause}}
    \begin{table}[h]
        \centering
        \begin{tabularx}{\textwidth}{|X|X|X|}
        \hline
        \multicolumn{1}{|c|}{neutre} & \begin{tabular}[c]{@{}l@{}}...\textcolor{orange}{parce que}...\\ ...\textcolor{orange}{car}...\end{tabular} & \begin{tabular}[c]{@{}l@{}}J'ai créé Famicity \textbf{parce que / car}\\ mon grand-père me l'a demandé.\end{tabular} \\ \hline
        \multicolumn{1}{|c|}{négative} & \begin{tabular}[c]{@{}l@{}}\textcolor{orange}{à cause de} + déterminant + nom\\ \textcolor{orange}{à cause de} + pronom\end{tabular} & \begin{tabular}[c]{@{}l@{}}Je suis en retard \textbf{à cause de} mon frère.\\ Je suis en retard \textbf{à cause de} lui.\end{tabular} \\ \hline
        \multicolumn{1}{|c|}{positive} & \begin{tabular}[c]{@{}l@{}}\textcolor{orange}{grâce à} + déterminant + nom\\ \textcolor{orange}{grâce à} + pronom\end{tabular} & \begin{tabular}[c]{@{}l@{}}C'est \textbf{grâce à} mon grand-père.\\ C'est \textbf{grâce à} lui.\end{tabular} \\ \hline
        \end{tabularx}
    \end{table}

    \begin{notebox}
    \textbf{Attention}
    \begin{enumerate}
        \item On ne peut pas utiliser \textit{car} en début de phrase.
        \item grâce a + le(s) $\rightarrow$ grâce au(x), à cause de + le(s) $\rightarrow$ à cause du/des \\
        On peut faire beaucoup de choses \textbf{grâce aux} nouvelles technologies. \\
        Je n'ai pas pu retrouver ma famille à la mer \textbf{à cause du} mauvais temps.
    \end{enumerate}
    \end{notebox}
    \item \textcolor{orange}{\textbf{La conséquence}}
    \begin{table}[h]
    \centering
    \begin{tabularx}{\textwidth}{|X|X|X|}
    \hline
    \multicolumn{1}{|c|}{\textcolor{orange}{donc, alors}} & \begin{tabular}[c]{@{}l@{}}indique une conséquence\end{tabular} & \begin{tabular}[c]{@{}l@{}}Je voulais \textbf{donc} lui permettre de\\ créer un arbre généalogique.\\ \textbf{Alors}, pour se donner des \\nouvelles, c’était l’idéal.\end{tabular} \\ \hline
    \multicolumn{1}{|c|}{\textcolor{orange}{c’est pourquoi}} & \begin{tabular}[c]{@{}l@{}}indique une conséquence\\ s’utilise surtout à l’écrit\end{tabular} & \begin{tabular}[c]{@{}l@{}}\textbf{C’est pourquoi} j’ai créé Famicity.\end{tabular} \\ \hline
    \end{tabularx}
    \end{table}

    \begin{notebox}
        C'est pourquoi s'utilise surtout à l’écrit.
    \end{notebox}
\end{enumerate}

\section{Les indéfinis de quantité}
Ces déterminants servent a exprimer une variation dans les quantités.
\begin{enumerate}
    \item \textcolor{orange}{\textbf{aucun(e)}}: quantité nulle \\
    e.g., Je \textbf{ne} connais \textbf{aucune} de ses œuvres.
    \begin{notebox}
        Aucun(e) est toujours au singulier et s'utilise avec ne.
    \end{notebox}
    \item \textcolor{orange}{\textbf{chaque}}: quantité unitaire. \\
    e.g., Une plante dans \textbf{chaque} pièce.
    \begin{notebox}
        Chaque est toujours au singulier.
    \end{notebox}
    \item \textcolor{orange}{\textbf{quelques}}: faible\footnote{adj. weak} quantité \\
    e.g., Je connais \textbf{quelques} créateurs.
    \begin{notebox}
        Quelques est toujours au pluriel.
    \end{notebox}
    \item \textcolor{orange}{\textbf{plusieurs}}: quantité plus importante \\
    e.g., J'ai assisté à \textbf{plusieurs} défiles.
    \begin{notebox}
        Plusieurs est toujours au pluriel.
    \end{notebox}
    \item \textcolor{orange}{tout(e), tou(te)s}: totalité \\
    e.g., \textbf{Tous} les créateurs étaient réunis.
    \begin{notebox}
    \begin{enumerate}
        \item Tout(e), tou(te)s s'accorde en genre et en nombre avec le nom qui suit.
        \item On ne prononce pas le \textit{s} de \textit{tous} quand il détermine un nom. \\
        e.g., Je n'ai pas acheté tou\underline{s} ces objets.
    \end{enumerate}
    \end{notebox}
\end{enumerate}

\section{La voix passive}
On utilise la voix passive pour mettre l'accent sur l'objet qui subit l'action, plutôt que sur le sujet qui fait l'action. Le règle est suivant:
\begin{center}
    Nouveau objet + \textcolor{orange}{être (conjugué)} + \textcolor{orange}{Participe Passé} + par + agent\footnote{qui a fait l'action}
\end{center}
Par exemple,
\begin{enumerate}
    \item Le chat mange la souris. $\rightarrow$ La souris est \textbf{mangée} par le chat.
    \item Il \textbf{a été créé} par Alphadi en 1998.
\end{enumerate}

\chapter{Unité 8}
\section{L’impératif et les pronoms indirects}
A l’impératif, pour remplacer une personne qui suit un verbe + \textit{à}, on utilise un pronom indirect. \\
e.g., Tu parles \underline{à} mon frère! $\rightarrow$ Parle-\textbf{lui}!
\begin{center}
    \textcolor{orange}{\textbf{moi, me*}} $\rightarrow$ je \\
    \textcolor{orange}{\textbf{toi, te*}} $\rightarrow$ tu \\
    \textcolor{orange}{\textbf{il, elle}} $\rightarrow$ il, elle \\
    \textcolor{orange}{\textbf{nous}} $\rightarrow$ nous \\
    \textcolor{orange}{\textbf{vous}} $\rightarrow$ vous \\
    \textcolor{orange}{\textbf{leur}} $\rightarrow$ ils, elles
\end{center}
\begin{notebox}
    \textbf{Attention}! À l’impératif affirmatif, les pronoms indirects \textit{me, te} $\rightarrow$ \textit{moi, toi}. \\
    e.g., Tu me téléphones. $\rightarrow$ Téléphone-moi!
\end{notebox}
\begin{enumerate}
    \item À l’impératif affirmatif, les pronoms compléments indirects \textcolor{orange}{\textbf{moi, toi, lui, nous, vous, leur}} se placent \textcolor{orange}{\textbf{après le verbe}} (avec un  tiret). \\
    e.g., Présentez-\textbf{lui} votre CV.
    \item À l’impératif négatif, les pronoms compléments indirects \textcolor{orange}{\textbf{me*, te*, lui, nous, vous, leur}} se placent \textcolor{orange}{\textbf{avant le verbe}}. \\
    Ne \textbf{me} téléphone pas! \\
    Ne \textbf{leur} rendez pas service.
    \begin{notebox}
        \textbf{Attention}
        \begin{enumerate}
            \item Devant une voyelle or un \textit{h}, me $\rightarrow$ m', te $\rightarrow$ t'.
            \item Le pronoms indirects est toujours \textit{à quelqu'un}. Pas \textit{à quelque chose}.
        \end{enumerate}
    \end{notebox}
\end{enumerate}

\section{Le but avec afin de et pour}
Quand on exprime un but, on répond a la question dans quel but? ou pour faire quoi?
\begin{enumerate}
    \item \textcolor{orange}{\textbf{pour}} + verbe a l'infinitif \\
    e.g., Elle fait un service civique \textbf{pour} \underline{faire} une action généreuse.
    \item \textcolor{orange}{\textbf{pour}} + pronom tonique \\
    e.g., Elle fait cela \textbf{pour} \underline{lui}.
    \item \textcolor{orange}{\textbf{pour}} + nom \\
    e.g., Elle fait ça \textbf{pour} \underline{l'avenir}\footnote{l'avenir: n.m. future}.
    \item \textcolor{orange}{\textbf{afin de}} + verbe a l'infinitif \\
    e.g., Il engage\footnote{engager: v. to appoint} un détective \textbf{afin de}\footnote{in order to} \underline{retrouver} son oncle.
\end{enumerate}

\section{Le conditionnel et la condition avec si}
\subsection{Le Conditionnel}
Le conditionnel sert à exprimer:
\begin{enumerate}
    \item la \textcolor{orange}{\textbf{politesse}} \\
    e.g., Je \textbf{voudrais} un café, s'il vous plaît.
    \item le \textcolor{orange}{\textbf{souhait}} \\
    e.g., J'\textbf{aimerais} vivre en Suisse.
    \item un \textcolor{orange}{\textbf{conseil}} \\
    e.g., Tu \textbf{devrais} travailler davantage.
    \item une \textcolor{orange}{\textbf{condition}} \\
    e.g., \underline{Si} j’étais, riche, j'\textbf{aurais} une piscine.
    \item Pour former le conditionnel, on utilise:
    \begin{center}
        infinitif du verbe + \textcolor{orange}{\textbf{ais, ais, ait, ions, iez, aient}}
    \end{center}
\end{enumerate}
\begin{notebox}
    \textbf{Attention} Le conditionnel suit les mêmes règles et exceptions que le futur simple. $\rightarrow$ Mon notes a la FIN!
\end{notebox}

\subsection{La condition}
La condition exprime un résultat possible.
\begin{enumerate}
    \item Pour une action réelle, on utilise:
    \begin{center}
        \textbf{si} + présente + présent ou futur
    \end{center}
    \textbf{Si} tu \textbf{veux} gagner de l'argent, tu \textbf{peux} travailler. \\
    \textbf{Si} j'\textbf{ai} assez d'argent, j'\textbf{achèterai} une voiture. \\
    \textbf{Si} j'\textbf{ai} assez d'argent, je \textbf{vais acheter} une voiture. (a l'oral)
    \begin{notebox}
        \textbf{Attention}
        \begin{enumerate}
            \item Quand l'action est un ordre ou un conseil, on utilise:
            \begin{center}
                si + présent + impératif
            \end{center}
            \textbf{Si} tu \textbf{veux} gagner de l'argent, \textbf{travaille}!
            \item Devant \textit{il} ou \textit{ils}. \textit{si} $\rightarrow$ \textit{s'} \\
            e.g., \textbf{S'}il neige, nous irons a la montagne.
        \end{enumerate}
    \end{notebox}
    \item Pour une action irréelle, on utilise:
    \begin{center}
        si + imparfait + conditionnel
    \end{center}
    \textbf{Si} j'\textbf{étais} plus jeune, je \textbf{ferais} le tour du monde a vélo. \\
    \textbf{Si} elle \textbf{était} riche, elle \textbf{achèterait} une maison.
\end{enumerate}

\chapter{Vocabulaire} 
\section{Unité 5}
\subsection*{Des qualités et des défauts}
\begin{multicols}{2}
    \entry{organisé}{}{adj.}{organized}{| It means plan for everything}

    \entry{solitaire}{}{adj.}{alone}{| travel alone}

    \entry{sociable}{}{adj.}{sociable}{| It means like to meet people, locals or friends during travel}

    \entry{inventif}{}{adj.}{creative}{| It means like to find new travel routes}

    \entry{curieux}{}{adj.}{curious}{}

    \entry{souriant}{}{adj.}{smiley}{}

    \entry{spontané}{}{adj.}{spontaneous}{| It means do things naturally, often guided by intuition}

    \entry{passionné}{}{adj.}{passionate}{}

    \entry{fier}{}{adj.}{proud}{}

    \entry{désorganisé/désordonné}{}{adj.}{unorganized}{}

    \entry{malhonnête}{}{adj.}{dishonest}{}

    \entry{maniaque}{}{adj.}{fastidious/obsessive}{| It means perfectionist, very paticular about details}

    \entry{têtu}{}{adj.}{stubborn}{| It means refuse to change their mind}

    \entry{maladroit}{}{adj.}{clumsy}{| It means lack coordination}
\end{multicols}
\subsection*{La forme physique}
\begin{multicols}{2}
    \entry{entretenir son corps}{}{}{maintain your body}{}

    \entry{grandir}{}{v.}{get taller}{| = s'allonger}

    \entry{faire de l'exercice}{}{}{exercise}{}

    \entry{un entraînement}{}{n.m.}{training}{}

    \entry{être sportif(ive)}{}{adj.}{sportive}{}

    \entry{se reposer}{}{v.}{rest, relax}{| = se détendre}
\end{multicols}
\subsection*{Partir en voyage}
\begin{multicols}{2}
    \entry{s’évader}{}{v.}{escape, get away}{}

    \entry{faire le tour du monde}{}{}{travel around the world}{}

    \entry{découvrir des paysages}{}{}{discover the scenary}{}

    \entry{préparer minutieusement}{}{prepare carefully}{}{}

    \entry{bronzer}{}{v.}{sunbath}{}

    \entry{se déplacer}{}{v.}{move, travel}{| \textbf{se déplacer} dans les paysages du monde entier}

    \entry{aller dans un endroit}{}{}{go to a place}{}
\end{multicols}
\subsection*{Le voyage}
\begin{multicols}{2}
    \entry{un voyageur}{}{n.m.}{traveller}{}

    \entry{un touriste}{}{n.m.}{tourist}{| le tourisme}

    \entry{un paysage}{}{n.m.}{scenary}{| la nature}

    \entry{une balade}{}{n.f.}{walk}{}

    \entry{le tourisme solidaire}{}{solidarity tourism}{| a trip to support the locals}

    \entry{un mode de vie nomade}{}{}{a nomadic\footnote{people who move from one place to another} lifestyle}{}
\end{multicols}
\subsection*{Se lancer un défi}
Launch a challenge/Challenge yourself
\begin{multicols}{2}
    \entry{supporter un rythme}{}{}{to keep up with a pace}{}

    \entry{constituer des équipes}{}{}{set up teams}{}

    \entry{s'affronter}{}{v.}{fight each other}{| Des milliers de salaries\footnote{salaries: employees} \textbf{s'affrontent} dans un challenge de marche au travail.}

    \entry{être dans le classement}{}{}{be in the ranking}{}

    \entry{consacrer du temps}{}{}{sacrifice the time}{}

    \entry{gagner}{}{v.}{to win}{| $\neq$ perdre\footnote{perdre: to lose}, main = être vainqueur\footnote{vainqueur: winner}}

    \entry{avoir peur de}{}{}{to be afraid of}{}

    \entry{se tester}{}{v.}{to try oneself out}{}
\end{multicols}
\subsection*{Miscs}
\begin{multicols}{2}
    \entry{rêve}{}{v.}{dream}{| rêve de quelque chose. }

    \entry{baigner}{}{v.}{walk around}{}

    \entry{être plongé}{}{}{be immersed in}{| \textbf{être plongé} dans l'image}

    \entry{se plonge dans}{}{}{to immerse oneself in}{| Je veux \textbf{me plonge dans} la nature.}

    \entry{se retrouver}{}{v.}{to find oneself}{| \textbf{se retrouver} dans des situations impossibles}

    \entry{A propos}{}{}{About us}{| Normalement, il s'agit d'une page qui contient\footnote{contenir: include} des informations sur nous sur le site.}

    \entry{Actus}{}{}{News}{| Page présentant les dernières actualités\footnote{actualités: news}, régulièrement mise à jour\footnote{mise à jour: update}.}

    \entry{s'inscrire}{}{v.}{to sign up for}{| \textbf{s'inscrire} à quelque chose}

    \entry{en direct}{}{adv.}{live}{}

    \entry{se battre}{}{v.}{to beat, fight}{| Ça signifie ``faire beaucoup d'efforts''. Les conjugaisons: Je \textbf{me} bats, tu \textbf{te} bats, il/elle/on \textbf{se} bat, nous \textbf{nous} battons, vous \textbf{vous} battez, ils/elles \textbf{se} battent}

    \entry{de temps en temps}{}{adv.}{from time to time}{}

    \entry{être un pissou}{}{}{a shit}{| = être peureux}
\end{multicols}
\subsection*{Les Phrases Utiles}
\begin{enumerate}
    \item J'aime apprendre de \textbf{me lancer de nouveaux défis}.\footnote{set myself on new challenges.}
    \item Nous devons \textbf{consacrer} deux heures et demie par jour \textbf{à} l'exercice.
    \item On \textbf{bosse} énormément = On travail énormément.
    \item \textbf{Encourager quelqu'un}
    \begin{itemize}
        \item Allez!
        \item Vas-y!
        \item Courage!
        \item Ne t’inquiète pas!
        \item Te peux \textbf{y} arriver!
        \item N'aie pas peur! N'ayez pas peur!
        \item Aie confiance! Ayez confiance!
    \end{itemize}
    \item \textbf{Insister pour inviter}
    \begin{itemize}
        \item Je compte sur toi!
        \item Mais si!
        \item Je t'assure.
        \item Tu dois absolument venir.
        \item Ça va être génial/super!
    \end{itemize}
\end{enumerate}

\section{Unité 6}
\subsection*{Les 5 sens}
\begin{multicols}{2}
    \entry{le goût}{}{n.m.}{taste}{| = goûter}

    \entry{l'odorat}{}{n.m.}{smell}{| = sentir}

    \entry{l’ouïe}{}{n.f.}{hearing}{| = ecouter/entendre}

    \entry{le toucher}{}{n.m.}{touch}{| = toucher}

    \entry{la vue}{}{n.f.}{sight}{| regarder/voir}

    \entry{une odeur}{}{n.f.}{smell}{}

    \entry{une saveur}{}{n.f.}{flavour}{}

    \entry{des sons}{}{n.m.}{sound}{| $\approx$ des bruits\footnote{bruit: noise}}
\end{multicols}
\subsection*{L'art et la culture}
\begin{multicols}{2}
    \entry{publier}{}{v.}{to publish}{| = sortir un album}

    \entry{une pochette d'album}{}{}{an album cover}{}

    \entry{une tournée}{}{n.f.}{tour}{}

    \entry{la scène}{}{n.f.}{stage}{| être sur scène\footnote{be on stage}}

    \entry{la classique}{}{n.f.}{classic}{| le jazz}

    \entry{une exposition}{}{n.f.}{exhibition}{| exposer\footnote{to exhibit}}

    \entry{une galerie}{}{n.f.}{gallery}{}

    \entry{l'art urbain}{}{n.m.}{urban art}{}
\end{multicols}
\subsection*{Les emotions}
\begin{multicols}{2}
    \entry{la joie}{}{n.f.}{joy}{}

    \entry{la tristesse}{}{n.f.}{sadness}{}

    \entry{la colère}{}{n.f.}{anger}{}

    \entry{la surprise}{}{n.f.}{surprise}{}

    \entry{le dégoût}{}{n.m.}{disgust}{}

    \entry{le désintérêt}{}{n.m.}{disinterest}{}

    \entry{la peur}{}{n.f.}{fear}{}
\end{multicols}
\subsection*{La nature}
\begin{multicols}{2}
    \entry{une feuille}{}{n.f.}{leaf}{}

    \entry{un arbre}{}{n.m.}{tree}{}

    \entry{une bête}{}{n.f.}{animal}{}

    \entry{un bois}{}{n.m.}{wood}{}

    \entry{le vent}{}{n.m.}{wind}{}
\end{multicols}
\subsection*{Des adjectifs pour apprécier}
\begin{multicols}{2}
    \entry{extraordinaire}{}{adj.}{extraordinary}{}

    \entry{impressionnant(e)}{}{adj.}{impressive}{}

    \entry{délicate(e)}{}{adj.}{delicate}{| En anglais, ça signifie ``fragile''}

    \entry{délicieux(euse)}{}{adj.}{delicious}{}

    \entry{drôle}{}{adj.}{funny}{}

    \entry{ironique}{}{adj.}{ironic}{}

    \entry{tendre}{}{adj.}{soft}{}

    \entry{original(e)}{}{adj.}{original}{}
\end{multicols}
\subsection*{Les adverbes}
\begin{multicols}{2}
    \entry{sainement}{}{adv.}{healthily}{| féminin + \textit{-ment}, e.g., Mange \textbf{sainement}}

    \entry{quasiment}{}{adv.}{nearly, almost}{| Cette pianiste de formation\footnote{formation: training} a composé et écrit son premier album, \textit{Brol}, \textbf{quasiment} seule.}

    \entry{immédiatement}{}{adv.}{immediately}{| féminin + \textit{-ment}}

    \entry{attentivement}{}{adv.}{attentively}{| féminin + \textit{-ment}}

    \entry{vraiment}{}{adv.}{really}{| voyelle, plus \textit{-ment} directement}

    \entry{profondément}{}{adv.}{deeply}{| féminin + \textit{-ment}. Mais c'est un cas particulier, parce que \textit{e} devient \textit{é}}

    \entry{suffisamment}{}{adv.}{sufficiently}{| \textit{-ant} (suffis\underline{ant})$\rightarrow$ \textit{-amment}}

    \entry{récemment}{}{adv.}{recently}{| \textit{-ent} (réc\underline{ent}) $\rightarrow$ \textit{-emment}}

    \entry{faussement}{}{adv.}{wronly}{| = incorrectement, féminin (fausse) + \textit{-ment}}

    \entry{puissamment}{}{adv.}{powerfully}{| \textit{-ant} (puiss\underline{ant}) $\rightarrow$ \textit{-amment}}

    \entry{prudemment}{}{adv.}{prudently, carefully}{| \textit{-ent} (prud\underline{ent}) $\rightarrow$ \textit{-emment}}

    \entry{certainement}{}{adv.}{certainly}{| féminin + \textit{-ment}}

    \entry{totalement}{}{adv.}{totally}{| féminin + \textit{-ment}}

    \entry{méchamment}{}{adv.}{badly}{| \textit{-ant} (méch\underline{ant}) $\rightarrow$ \textit{-amment}}

    \entry{régulièrement}{}{adv.}{regularly}{| féminin + \textit{-ment}}

    \entry{rapidement}{}{adv.}{rapidly}{| féminin + \textit{-ment}}
\end{multicols}
\subsection*{Miscs}
\begin{multicols}{2}
    \entry{la façon}{}{n.f.}{way}{| la meilleure \textbf{façon} de l’infinitif}

    \entry{dispose de}{}{v.}{have sth available to us}{| Nous \textbf{disposons} de cinq sens.}

    \entry{à partir de}{}{prep.}{from}{| Nous construisons donc notre monde \textbf{à partir de} 20\% de toutes les informations disponibles.\footnote{So we build our world from 20\% of all available information.}}

    \entry{se servir de}{}{v.}{use}{| Le plus difficile d'apprendre à \textbf{se servir de} nos sens.}

    \entry{à force de}{}{}{by, thanks to}{| \textbf{À force d'}écouter tous ces sons}

    \entry{les renseignements}{}{n.m.}{information}{| = les informations}

    \entry{s'asseoir}{}{v.}{to sit down}{| Je m'ass\textbf{ie}ds, tu t'ass\textbf{ie}ds, il/elle/on s'ass\textbf{ie}d, nous nous asse\textbf{y}ons, vous vous asse\textbf{y}ez, ils/elles s'ass\textbf{ey}ent}

    \entry{un album très attendu}{}{}{a highly anticipated album}{}

    \entry{Autre atout}{}{}{Another strength}{| \textbf{Autre atout}: sa créativité.}

    \entry{une pause gourmande}{}{}{a food break}{| faire \textbf{une pause gourmande}}

    \entry{prendre soin de quelqu'un}{}{}{take care of}{| Est-ce que tu penses à \textbf{prendre soin de} toi?}

    \entry{bouger}{}{v.}{to move}{| \textbf{Bouge} ton corps!}

    \entry{les pays de rêves}{}{}{the land of dream}{}

    \entry{agir}{}{v.}{to act}{}

    \entry{attraper l'argent}{}{}{catch the money}{| Ça signifie ``gagner sa vie''}
\end{multicols}
\subsection*{Les Phrases Utiles}
\begin{enumerate}
    \item \textbf{Donner un avis/une appréciation}
    \begin{itemize}
        \item A mon avis, $\cdots$
        \item Je pense que $\cdots$
        \item Je trouve que $\cdots$
        \item C'est génial !
        \item Pour moi, $\cdots$
        \item Bof !\footnote{Bof: meh}
        \item Je n'ai pas vraiment aimé.
        \item Je suis déçu(e).\footnote{I am disappointed }
    \end{itemize}
    \item \textbf{Pour décrire une sensation}
    \begin{itemize}
        \item J'aime beaucoup marcher pied nus\footnote{nus: naked} sur le sable.
        \item Je \textbf{sens} sa chaleur\footnote{chaleur: heat} sous mes pieds.
        \item J'\textbf{entends} le son relaxant de l’océan et je \textbf{vois} le beau ciel bleu à l'horizon.
        \item J'adore \textbf{goûter} de nouvelles saveurs\footnote{saveur: flavour}!
    \end{itemize}
    \item \textbf{Ce que} vs. \textbf{Ce qui}
    \begin{itemize}
        \item \textbf{Ce que}: Peut être \textbf{COD} et représente une chose ou une idée abstraite. Il répond à ``quoi ?'' après un verbe.\\
        e.g., J'aime \textit{quoi}? $\rightarrow$ J'aime \textbf{ce que} tu cuisines.
        \item \textbf{Ce qui}: Quand ``ce qui'' est utilisé, il est généralement le \textbf{sujet} de la proposition subordonnée. Il représente une chose ou une idée abstraite\footnote{abstraite: abstract} qui fait l'action du verbe dans cette subordonnée. \\
        e.g., \textit{Quoi} est surprenant\footnote{surprenant: surprising}? $\rightarrow$ \textbf{Ce qui} arrive est surprenant.
    \end{itemize}
    \item \textbf{Pour exprimer ton sensation}
    \begin{itemize}
        \item \textbf{ressentir de + nom}: e.g., Je \textbf{ressens de} la tristesse.
        \item \textbf{sentir + adjectif}: Pour decrire une odeur ou une impression extérieure\footnote{extérieure: external}. e.g., Ça \textbf{sent} mauvais.
        \item \textbf{se sentir + adjectif}: Pour decrire un état\footnote{état: state} physique ou émotionnel interne\footnote{interne: internal}. e.g., Je \textbf{me sens} bien.
    \end{itemize}
    \item \textbf{Acheter}
    \begin{itemize}
        \item Je voudrais/souhaiterais acheter $\cdots$
        \item Est-ce que vous auriez\footnote{C'est conditionnel présent ici.} $\cdots$?\footnote{Ça signifie: Would you have $\cdots$?}
        \item Je peux payer par carte bancaire / par chèque / en espèces?
        \item Je voudrais une facture\footnote{facture: receipt}.
        \item Nous avons un budget précis.
    \end{itemize}
    \item \textbf{Conseiller}
    \begin{itemize}
        \item \textbf{Quelqu'un a bien fait de + l'infinitif}: e.g., Tu \textbf{as bien fait} de m’écrire.
        \item \textbf{conseiller à quelqu'un de faire quelque chose}
        \item Je te / vous conseille de $\cdots$
        \item Il faut $\cdots$
        \item Il est utile / inutile\footnote{inutile: useless} de $\cdots$
        \item Tu devrais / Vous devriez\footnote{You should $\cdots$} $\cdots$
        \item Tu pourrais / Vous pourriez\footnote{You cound $\cdots$} $\cdots$
        \item Ce serait mieux de\footnote{It would be better to $\cdots$} $\cdots$
        \begin{notebox}
            \textbf{Attention !} Ce serait mieux de réviser cette section parce que peut-être c'est le sujet de votre composition!
        \end{notebox}
    \end{itemize}
\end{enumerate}
\section{Unité 7}
\subsection{Les participes passe irréguliers}
\begin{multicols}{2}
    \entry{venir}{}{}{venu}{}

    \entry{faire}{}{}{fait}{}

    \entry{voir}{}{}{vu}{}

    \entry{prendre}{}{}{pris}{}

    \entry{être}{}{}{été}{}

    \entry{avoir}{}{}{au}{}

    \entry{boire}{}{}{bu}{}

    \entry{lire}{}{}{lu}{}

    \entry{entendre}{}{}{entendu}{}

    \entry{offrir}{}{}{offert}{}

    \entry{connaître}{}{}{connu}{}

    \entry{grandir}{}{}{grandi}{}

    \entry{ouvrir}{}{}{ouvert}{}

    \entry{mourir}{}{}{mort}{}

    \entry{savoir}{}{}{su}{}
\end{multicols}
\subsection{L'accessoire}
\begin{multicols}{2}
    \entry{la tresse}{}{n.f.}{braid}{| Ça signifie ``long hair'' en anglais}

    \entry{la coiffure}{}{n.f.}{hairstyle}{}

    \entry{un tatouage}{}{n.m.}{tatoo}{}

    \entry{un henné}{}{n.m.}{the makeup on nail}{}

    \entry{le maquillage}{}{n.m.}{makeup}{}

    \entry{le bijoux}{}{n.m.}{jewellery}{}

    \entry{un foulard}{}{n.m.}{scarf}{}
\end{multicols}
\subsection{La mode}
\begin{multicols}{2}
    \entry{un défile}{}{n.m.}{a parade}{}

    \entry{la haute couture}{}{n.f.}{high-fashion}{}

    \entry{une marque}{}{n.f.}{brand}{}

    \entry{un styliste}{}{n.}{designer}{}

    \entry{classique}{}{adj.}{classic}{| = tranditionnel, $\neq$moderne}

    \entry{original}{}{adj.}{original}{| $\neq$ commun}
\end{multicols}
\subsection{Miscs}
\begin{multicols}{2}
    \entry{un parcours}{}{n.m.}{route}{}

    \entry{sur}{}{adj.}{safe}{}

    \entry{privé}{}{adj.}{private}{| un lieu de partage sur et \textbf{privé} sur Internet.}

    \entry{chacun}{}{pron.}{each one}{| \textbf{Chacun} a son travail.}

    \entry{retrouver}{}{v.}{to find}{| Vous l'avez déjà trouvé.}

    \entry{chercher}{}{v.}{to search}{| Vous ne l'avez pas encore trouvé.}

    \entry{un bout}{}{n.m.}{tip}{| un \textbf{bout} de\footnote{a piece of}}

    \entry{la verdure}{}{n.f.}{greenary}{}

    \entry{proche}{}{n.m.f./adj}{close friends}{}

    \entry{chiner}{}{v.}{to bargain hunt}{| chercher des objets d'occasion.}

    \entry{retro}{}{adj.}{retro}{| d'un style ancien}

    \entry{si}{}{conj./adv.}{so}{| Tu es \textbf{si} grande maintenant !}

    \entry{rassembler}{}{v.}{to gather}{}

    \entry{valoriser}{}{v.}{to add value to}{}

    \entry{la étagère}{}{adj.}{shelf}{}

    \entry{charger}{}{v.}{to load}{| comme ``telecharger: to download''}

    \entry{le génie}{}{n.m.}{genius}{}

    \entry{maléfique}{}{adj.}{evil}{}

    \entry{empêcher}{}{v.}{to prevent}{}
\end{multicols}
\subsection{Les Phrases Utiles}
\begin{enumerate}
    \item permettre \textbf{à} quelqu'un \textbf{de faire} quelque chose\\
    e.g., On veut \textbf{permettre} à chacun de pouvoir retrouver à tout moment\footnote{at any time} les membres de sa famille.
    \item tous les deux années\footnote{every two years}
    \item éclairer la lanterne\footnote{ça signifie ``faire comprendre, expliquer quelque chose à quelqu'un''.}
    \item Des phrases sur \textbf{rappeler}
    \begin{itemize}
        \item \textbf{Rappeler quelqu'un}: appeler quelqu'un qui a déjà téléphoné.
        \begin{itemize}
            \item rappeler à quelqu'un de faire: \textbf{Remind} someone to do.
        \end{itemize}
        \item \textbf{Se rappeler quelqu'un}: se souvenir de quelqu'un.
        \item \textbf{Se souvenir de quelque chose}: penser à quelque chose du passé.
        \item \textbf{Se rappeler quelque chose}: parler de quelque chose qu'on avait oublié.
        \item \textbf{rappeler à l'ordre}: donner un avertissement.
    \end{itemize}
    \item \textbf{La famille}
    \begin{itemize}
        \item \textbf{une arbre généalogique}: a family tree.
        \item \textbf{le PACS}: A civil relationship other than marriage.
        \item \textbf{tenir de quelqu'un}: to care about someone, to be deeply attached to someone.
        \item \textbf{ressembler à quelqu'un}: to look like someone
    \end{itemize}
    \item \textbf{Parler d'un changement}
    \begin{itemize}
        \item \textbf{changer de voie}: Change the job.
        \item \textbf{remettre au goût du jour}: Bring something up to date.
        \item \textbf{propose une vision de}: Propose a vision of $\cdots$
    \end{itemize}
    \item \textbf{Décrire un objet}
    \begin{itemize}
        \item \entry{personnalise}{}{adj.}{personalised}{}
        \item \textbf{personnalise}: C'est un adjectif et ça veut dire ``personalised'' en anglais.
        \item \textbf{en soie}: in silk
        \item \textbf{en porcelaine}: in china
        \item \textbf{en bois}: in wood
        \item \textbf{C'est (très) utile pour}: It is very useful for $\cdots$
    \end{itemize}
    \item \textbf{Retrouver}
    \begin{itemize}
        \item \textbf{donner des nouvelles}: give some news.
        \item \textbf{prendre des nouvelles}: ask for some news. C'est l'opposite de ``donner des nouvelles''.
        \entry{se souvenir}{}{v.}{to remember}{}
        \item \textbf{replonger dans le passé}: dive back into the past
        \entry{hériter}{}{v.}{to inherit}{}
    \end{itemize}
    \item \textbf{Prendre des nouvelles}
    \begin{itemize}
        \item Vous allez bien? Comment ça va? Tu vas mieux?
        \item Qu'est ce-que tu deviens?
        \item Je t'appelle pour prendre de te tes nouvelles.\footnote{I am calling to check on you.}
    \end{itemize}
    \item \textbf{Donner des nouvelles}
    \begin{itemize}
        \item J'ai une bonne / mauvaise nouvelle.
        \item Je suis toujours $\cdots$
        \item Je vais $\cdots$
        \item Je suis en \textbf{congé}\footnote{n.m. leave}
        \item J'ai eu un accident.
        \item Elle va mieux.
        \item Il est guéri\footnote{healed}
    \end{itemize}
    \item \textbf{Raconter une histoire}
    \begin{itemize}
        \item Il était une fois $\cdots$
        \item C’était en $\cdots$
        \item Un jour, le lendemain\footnote{the next day}, la veille\footnote{the day before}
        \item Au début
        \item Ensuite, puis, finalement
        \item Alors, donc, ainsi\footnote{In this way}
        \item Soudain\footnote{adv. suddenly}
    \end{itemize}
\end{enumerate}
 \subsection{La culture}
\begin{enumerate}
    \item \textbf{Le FIMA}
    \begin{itemize}
        \item \textbf{Quoi}: Le FIMA est le Festival International de la Mode en Afrique.
        \item \textbf{Où}: C'est en Afrique. Il a été créé au Niger.
        \item \textbf{Qui l'a créé}: Il a été créé par Alphadi, un créateur de mode nigérien.
        \item \textbf{Quand}: Le festival a été créé en 1998.
        \item \textbf{Pour qui}: Pour les créateurs de mode africains, les artistes et les journalistes du monde entier.
    \end{itemize}
    \item \textbf{Makembé}
    \begin{itemize}
        \item \textbf{Qui}: Makembé est le héros qui combat un génie maléfique pour sauver son village.
        \item \textbf{Quel pays}: C'est un conte de la République Démocratique du \textbf{Congo}.
        \item \textbf{Où}: L'histoire se passe dans un village et sur une montagne.
        \item \textbf{Qu'est-ce qui s'est passé}: Makembé cherche un arc magique et utilise une poudre magique pour vaincre (defeat) le génie.
        \item \textbf{La fin}: Le génie est tombé et les enfants peuvent naître dans le village.
    \end{itemize}
\end{enumerate}
\section{Unité 8}
\subsection{Améliorer un logement}
\begin{multicols}{2}
    \entry{améliorer}{}{v.}{to improve}{}

    \entry{un logement}{}{n.m.}{accommodation}{}

    \entry{le meuble}{}{n.m.}{furniture}{}

    \entry{locataire}{}{n.m./n.f.}{tenant\footnote{a person who occupies land or property rented from a landlord.}}{}

    \entry{le rangement}{}{n.m.}{storage}{}

    \entry{le tiroir}{}{n.m.}{drawer}{}

    \entry{le couchage}{}{n.m.}{sleeping unit}{}

    \entry{accéder}{}{v.}{to access/reach}{}

    \entry{rénover}{}{v.}{to renovate}{| une rénovation}

    \entry{bricoler}{}{v.}{to DIY}{| la bricolage}

    \entry{un chantier}{}{n.m.}{construction site}{}

    \entry{faire des travaux}{}{}{to do work/renovation}{}

    \entry{aménager}{}{v.}{to set up}{| \textbf{aménager} la chambre, ça  veut dire il y a rien dans la chambre au début.}

    \entry{isoler}{}{v.}{insulate}{}

    \entry{ventiler une pièce}{}{}{ventilate a room}{}

    \entry{un bâtiment basse}{}{}{a low building}{}

    \entry{la consommation}{}{}{consumption}{}
\end{multicols}
\subsection{Exprimer un souhait}
\begin{multicols}{2}
    \entry{tant de}{}{adv.}{so much}{| Après \textbf{tant d}’année de silence.}

    \entry{à peu près}{}{adv.}{around}{| Elle a \textbf{à peu près} ton âge.}

    \entry{notaire}{}{n.m./n.f.}{lawyer}{}

    \entry{une dispute}{}{n.f.}{dispute}{| un désaccord entre deux personnes.}

    \entry{un testament}{}{n.m.}{last will}{| un texte avec les souhaits\footnote{n.m. wish} d'une personne après sa mort\footnote{n.f. death}.}

    \entry{bonheur}{}{n.m.}{joy}{}

    \entry{des pièces jointes}{}{n.f.}{attachments}{}

    \entry{la voie publique}{}{n.f.}{public space}{}

    \entry{sale}{}{adj.}{dirty}{}
\end{multicols}
\subsection{S'engager}
\begin{multicols}{2}
    \entry{un service civique}{}{n.m.}{civic service\footnote{a voluntary engagement in a community service}}{}

    \entry{un volontaire}{}{n.m./adj.}{volunteer/voluntary}{}

    \entry{avoir l'occasion de}{}{}{have the change to do}{}

    \entry{une mission}{}{n.f.}{task}{}

    \entry{un partenariat}{}{n.m.}{partnership}{}

    \entry{être au service de qqn/qch}{}{}{to be in service of serving sb/sth\footnote{be ready to help with something or someone}}{}
\end{multicols}
\subsection{Des objets de la maison}
\begin{multicols}{2}
    \entry{un micro-onde}{}{n.m.}{microwave}{}

    \entry{des fenêtres}{}{n.f.}{window}{}

    \entry{des volets}{}{n.m.}{shutter\footnote{the window block, c'est commun en France}}{}

    \entry{une douche}{}{n.f.}{shower stall}{}

    \entry{un canapé convertible}{}{n.m.}{a convertable sofa}{}

    \entry{le chauffage}{}{n.m.}{heater}{}

    \entry{un mur}{}{n.m.}{wall}{}

    \entry{un escalier}{}{n.m.}{the staircase}{}

    \entry{une marche}{}{n.m.}{a step}{| C'est un parte de un escalier}
\end{multicols}
\subsection{Rendre service}
\begin{multicols}{2}
    \entry{aider}{}{v.}{to help}{| \textbf{aider} gracieusement}

    \entry{faire une action généreuse}{}{}{do a generous action}{}

    \entry{ouvrir sa porte}{}{}{open his door}{}

    \entry{proposer ses services}{}{}{offer your services}{}

    \entry{créer du lien social}{}{}{create social connections}{}
\end{multicols}
\subsection{Faire des démarches administratives}
\begin{multicols}{2}
    \entry{accéder a un service}{}{}{access a service}{}

    \entry{donner accès à}{}{}{give access t}{}

    \entry{parvenir à}{}{}{achieve}{}

    \entry{contacter un notaire}{}{}{contact a notary}{}

    \entry{rédiger un testament}{}{}{write a testament}{}

    \entry{hériter}{}{v.}{to inherit}{}

    \entry{recevoir un héritage}{}{}{receive an inheritance}{}

    \entry{engager un détective}{}{}{appoint a detective}{}
\end{multicols}
\subsection{Le logement}
\begin{multicols}{2}
    \entry{l'habitat}{}{n.m.}{accommodation}{}

    \entry{un propriétaire}{}{n.m.}{landlord/owner}{}

    \entry{un bureau}{}{n.m.}{desk}{}
\end{multicols}
\subsection{Les Phrases Utiles}
\begin{enumerate}
    \item \textbf{rendre service à quelque'un}: do someone a favor.
    \item \textbf{Des gens dans le besoin}: The people in need.
    \item quelque chose \textbf{occupe}\footnote{v. to equip} quelque chose = quelque chose \textbf{est munie de/est équipé de\footnote{be equipped with}} quelque chose
    \item parvenir à quelque chose\footnote{achieve something}
    \item Exprimer son mécontentement
    \begin{itemize}
        \item C'est n'importe quoi\footnote{That's ridiculous}!
        \item Ce n'est pas normal!
        \item J'en ai assez! J'en ai marre!\footnote{I've had enough! I'm fed up!}
        \item Ça m’énerve!\footnote{That annoys me!}
        \item C'est insupportable.\footnote{It's unbearable!}
        \item Ce n'est pas possible.
    \end{itemize}
    \item \textbf{Recommander un logement}
    \begin{itemize}
        \item Je recommande fortement\footnote{highly} ce logement.
        \item Je conseille vivement. (La même signification que ci-dessus\footnote{above})
        \item L’équipement est complet.
        \item La chambre est propre.
        \item Il ne manque rien.\footnote{Nothing is missing}
        \item Je reviendrai avec plaisir.\footnote{I will come back with pleasure.}
        \item Vous pouvez y aller les yeux fermes! (Ça veut dire quelque chose est très simple!)
    \end{itemize}
\end{enumerate}
\end{document}