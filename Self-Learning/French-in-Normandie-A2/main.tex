\documentclass[math,code]{amznotes}
\setcounter{tocdepth}{2}  % Show subsections, sections and chapters in the ToC
\usepackage[utf8]{inputenc}
\usepackage{amsmath}
\usepackage{amsfonts}
\usepackage{graphicx}
\usepackage{tikz}
\usepackage{etoolbox}
\usepackage{tabularx}
\usepackage{float} % Needed for [H] placement specifier
\usepackage{wrapfig} % Needed for wrapping figures
\usepackage{diagbox} % For diagonal split in table
\usepackage{booktabs} % For better table formatting
\usepackage[normalem]{ulem} % For Strikethrough formatting
\usepackage{multirow} % For multirow
\usepackage{xeCJK} % For Chinese support
\usepackage{longtable} % For longtable

\graphicspath{ {./images/} }
\geometry{
    a4paper,
    headheight = 1.5cm
}

\patchcmd{\chapter}{\thispagestyle{plain}}{\thispagestyle{fancy}}{}{}

\theoremstyle{remark}
\newtheorem*{claim}{Claim}
\newtheorem*{remark}{Remark}
\newtheorem{case}{Case}

\newcommand{\entry}[5]{\markboth{#1}{#1}\textbf{#1}\ {(#2)}\ \textit{#3}\ $\bullet$\ \textbf{\textit{#4}} \begin{small}#5\end{small}}  % Defines the command to print each word on the page, \markboth{}{} prints the first word on the page in the top left header and the last word in the top right

%----------------------------------------------------------------------------------------

\begin{document}
\fancyhead[L]{
    French in Normandie
}
\fancyhead[R]{
    Learning Notes
}
\tableofcontents

\chapter{La Semaine 1}
\section{Leçon 1}
\subsection{Vocabulaire}
\subsubsection*{Les Dimensions}
\begin{multicols}{2}
    \entry{minuscule}{}{adj.}{tiny}{| $\neq$ géant = énorme\footnote{huge}}

    \entry{haut}{}{adj.}{high}{| $\neq$ bas\footnote{low}}

    \entry{long}{}{adj.}{long}{| $\neq$ court\footnote{short}}

    \entry{étroit}{}{adj.}{narrow}{| $\neq$ large\footnote{wide}}

    \entry{mince}{}{adj.}{small}{| $\neq$ gros\footnote{big}}

    \entry{lourd}{}{adj.}{heavy}{| $\neq$ léger\footnote{light}}

    \entry{moyenne}{}{adj.}{average}{| de taille moyenne\footnote{medium-sized}}

    \entry{épais}{}{adj.}{thick}{}
\end{multicols}

\subsubsection*{La matière}
\begin{notebox}
    \begin{remark}
        Pour utiliser la matière, nous utilisons ``\textbf{en $+$ la matière}''. Par exemple, on dit que c'est une chose en bois.\footnote{It's a wooden thing.}''
    \end{remark}
\end{notebox}
\begin{multicols}{2}
    \entry{coton}{}{n.m.}{cotton}{}

    \entry{laine}{}{n.f.}{wool}{}

    \entry{tissu}{}{n.m.}{fabric}{}

    \entry{soie}{}{n.f.}{silk}{}

    \entry{verre}{}{n.m.}{glass}{}

    \entry{faïence}{}{n.f.}{china}{}

    \entry{argent}{}{n.m.}{silver}{}

    \entry{or}{}{n.m.}{gold}{}
\end{multicols}

\subsubsection*{L'utilité}
\begin{notebox}
    \begin{remark}
        This structure is used to express that something can be used for doing what, \textit{tout + infinitif}
    \end{remark}
\end{notebox}
\begin{multicols}{2}
    \entry{permettre de}{}{}{help, allow to do}{| ça \textbf{permet} de protéger $\cdots$ \footnote{It helps to protect $\cdots$}}

    \entry{servir à}{}{}{serve to do}{| C'est un chose qui \textbf{sert} à allumer une cigarette.}

    \entry{être pratique pour}{}{}{be practical for}{}

    \entry{être utile pour}{}{}{be useful for}{}

    \entry{être indispensable pour}{}{}{be essential for}{}
\end{multicols}

\subsubsection*{Misc}
\begin{multicols}{2}
    \entry{connu}{}{adj.}{well-known}{| C'est \textbf{connu} en tout le monde.\footnote{Everybody knows it.}}
    
    \entry{Normalement}{}{adv.}{Usually}{| \textbf{Normalement}, les gens qui utilisent ce sac sont des personnes qui aiment la mode}
    
    \entry{D'autre part}{}{adv.}{On the other hand}{}
    
    \entry{chic}{}{adj.}{beautiful}{| Louis Vuitton est très cher mais bien \textbf{chic}.}

    \entry{domaine}{}{n.m.}{field}{| Ils sont les plus beaux dans ce \textbf{domaine}.}

    \entry{penser à}{}{}{think of}{| Quand vous \textbf{pensez aux} marques francaises.\footnote{When you think of the French brands}}

    \entry{oublier}{}{v.}{forget}{| vous ne pouvez pas \textbf{oublier} les vins.\footnote{You cannot forget the wine.}}

    \entry{comme}{}{conj.}{like}{| Nous pensons que les produits en supermarché sont toujours locaux \textbf{comme} les produits laitiers.\footnote{We believe that supermarket products are always local, like dairy products.}}

    \entry{étui}{}{n.m.}{case}{| un \textbf{étui} à lunettes}

    \entry{appareil}{}{n.m.}{device}{}

    \entry{arroser}{}{v.}{water}{| \textbf{arroser} les plantes}

    \entry{pouvoir}{}{v.}{enable}{| pour \textbf{pouvoir} brancher\footnote{to be able to connect}}

    \entry{intérieur}{}{n.m.}{inside}{| dans l'intérieur}

    \entry{plusieurs}{}{adj.}{several}{}
\end{multicols}

\subsection{Phrase}
\begin{enumerate}
    \item Mise en route.\footnote{Getting Started.}
    \item Si on dit ``marques françaises'' à quoi cela vous fait penser ?\footnote{When we say “French brands,” what does that make you think of?}
    \item Les produits \textbf{tiennent longtemps}.\footnote{The product lasts a long time.}
    \item Les touristes peuvent apporter une petite partie de Paris.\footnote{Tourists can bring a little piece of Paris.}
    \item Elle mesure 20 centimètres de long et 8 centimètres de large.\footnote{It measures 20 centimeters long and 8 centimeters wide.}
\end{enumerate}

\section{Leçon 2}
\subsection{Vocabulaire}
\subsubsection*{Les vêtements}
\begin{multicols}{2}
    \entry{pull}{}{n.m.}{sweater}{}

    \entry{le sac à main}{}{}{handbag}{}

    \entry{la boucle d'oreille}{}{}{the earing}{}

    \entry{l'écharpe}{}{n.f.}{scarf}{}

    \entry{les gants}{}{n.m.}{gloves}{}

    \entry{tenue}{}{n.f.}{outfit}{| decrire la tenue d'une personne}
\end{multicols}
\subsubsection*{Miscs}
\begin{multicols}{2}
    \entry{coloré}{}{adj.}{colorful}{}

    \entry{motif}{}{n.m.}{pattern}{| Normalement, on utilise \textbf{à + motif}, par exemple, \textbf{à des fleurs}}

    \entry{suivre la mode}{}{}{poter ce qui est à la mode}{| Elle aime \textbf{suivre la mode}.}

    \entry{être a la mode}{}{}{être habiller comme les gens modernes}{| Ce jean est \textbf{à la mode}.}

    \entry{être tendance}{}{}{être moderne et populaire}{| Cette couleur \textbf{est tendance}.}

    \entry{sortir de la mode}{}{}{ne plus \textbf{être à la mode}}{}

    \entry{être démode}{}{}{ne plus \textbf{être moderne}}{}

    \entry{les tendances évoluent}{}{}{les styles changent avec le temps}{}

    \entry{une queue de cheval}{}{}{a ponytail}{}

    \entry{passer un examen}{}{}{take an exam}{}

    \entry{réussir un examen}{}{}{pass an exam}{}
\end{multicols}

\subsection{Phrase}
\begin{enumerate}
    \item \textbf{Porter + vêtements, en + matière, à + motifs} \\
    e.g., Elle porte un pantalon à carreaux en velours.\footnote{She is wearing velvet plaid pants.}
    \item Une chemise verte \textbf{à manches longues}.\footnote{A green long-sleeved shirt}
\end{enumerate}

\section{Leçon 3}
\subsection{Vocabulaire}
\begin{multicols}{2}
    \entry{se rendre compte que}{}{}{to realize that}{| On \textbf{se rend} vite \textbf{compte que} la publicité est partout\footnote{\textbf{adv.} everywhere}.}

    \entry{attrayante}{}{adj.}{attractive}{| des images \textbf{attrayantes}}

    \entry{efficace}{}{adj.}{effective}{| des slogans \textbf{efficaces}}

    \entry{car}{}{conj.}{because}{| La publicité est utile pour les marques, \textbf{car} elle leur permet de vendre plus.\footnote{because it allows them to sell more.}}

    \entry{grâce à}{}{prép}{thanks to}{| L'état gagne de l'argent \textbf{grâce aux} taxes sur les ventes.\footnote{The state earns money thanks to the sales taxes.}}

    \entry{se plaindre}{}{v.}{to complain}{| De nombreuses personnes \textbf{se plaignent de} cette pollution visuelle.\footnote{Many people complain about this visual pollution.}}

    \entry{empêcher de}{}{v.}{to prevent}{| Elles \textbf{empêchent} parfois \textbf{de} profiter du paysage.\footnote{They sometimes prevent you from enjoying the scenery.}}

    \entry{cible}{}{n.f.}{target}{| Les premières cibles}

    \entry{convaincant}{}{adj.}{convincing}{}

    \entry{Petit à petit}{}{adv.}{Gradually}{}

    \entry{disparu}{}{adj.}{disappeared}{}

    \entry{paysage}{}{n.m.}{scenary}{}

    \entry{étoile}{}{n.f.}{star}{| le ciel \textbf{étoilé}\footnote{\textbf{adj.} full of stars}}
\end{multicols}

\subsection{Phrase}
\begin{enumerate}
    \item C'est \textbf{claire et facile à} comprendre.\footnote{It's clear and easy to understand.}

    \item Une bonne pub \textbf{doit être} ...\footnote{A good publicity has to be ...}
\end{enumerate}

\section{Leçon 4}
\subsection{Vocabulaire}
\begin{multicols}{2}
    \entry{une boisson gazeuse}{}{n.f.}{soft drink}{}

    \entry{parfum}{}{n.m.}{perfume}{}

    \entry{sentir}{}{v.}{feel}{| Vous pouvez \textbf{sentir} Paris.}

    \entry{mettre}{}{v.}{to put on}{| Si vous \textbf{mettez} du parfum, }

    \entry{promouvoir}{}{v.}{to promote}{| La pub \textbf{promeut} la consommation du lait.}

    \entry{santé}{}{n.f.}{health}{}
\end{multicols}

\subsection{Phrase}
\begin{enumerate}
    \item Cette publicité \textbf{s'adresse / est adressée aux} enfants.\footnote{That publicity is aimed at kids.}
\end{enumerate}

\section{Grammaire}
\subsection{Hypothèse}
On fait une hypothèse sur le présent lorsqu'on imagine une chose, un fait qui n'existe pas ou qui n'est pas vrai au moment ou on le dit.\footnote{We make an assumption about the present when we imagine something, a fact that does not exist or is not true at the time we say it.} \\
\centerline{\color{orange}Si + imparfait, + conditionnel présent}
Par exemple,
\begin{enumerate}
    \item Si j'étais grand, je ferais du basketball. (Mais en fait, je ne suis pas grand $\cdots$)
    \item Elles ne vont pas au théâtre parce qu'elles n'ont pas le temps. $\Leftrightarrow$ Si elles avaient le temps, elles iraient au théâtre.
\end{enumerate}
\begin{notebox}
    \begin{remark}
        Notez que dans ce cas, l'imparfait n'a pas de valeur passée. Il faut considérer l'imparfait dans sa valeur modale. L'imparfait ici = \textbf{présent imaginaire}\footnote{Note that in this case, the imperfect tense has no past value. We must consider the imperfect tense in its modal value. The imperfect tense here = \textbf{imaginary present}}
    \end{remark}
\end{notebox}

\subsection{L'imparfait} 
L'imparfait est un temps qui exprime le passé. C'est un temps qui exprime une durée indéfinie et qui nous sert à expliquer le décor de l'action, la répétition dans le passé, etc.\footnote{It is a tense that expresses an indefinite duration and which helps us explain the setting of the action, repetition in the past, etc.}

Pour la plupart des verbes, le radical de l'imparfait est la première personne du pluriel (\textit{nous}) du présent de l'indicatif.\footnote{For most verbs, the root of the imperfect tense is the first person plural (\textit{nous}) of the present indicative.}

\begin{center}
\begin{tabular}{>{\bfseries}l c}
\toprule
Je & -ais \\
Tu & -ais \\
Il/Elle/On & -ait \\
Nous & -ions \\
Vous & -iez \\
Ils/Elles & -aient \\
\bottomrule
\end{tabular}
\end{center}

\begin{center}
\begin{tabular}{|l|l|l|}
\hline
\textbf{Verbe} & \textbf{Présent} & \textbf{Imparfait} \\
\hline
\textbf{Aimer} & Nous \textbf{aim}ons & J'aim\textbf{ais} \\
\hline
\textbf{Choisir} & Nous \textbf{choisiss}ons & Tu choisiss\textbf{ais} \\
\hline
\textbf{Partir} & Nous \textbf{part}ons & Il part\textbf{ait} \\
\hline
\textbf{Pouvoir} & Nous \textbf{pouv}ons & Nous pou\textbf{}v\textbf{ions} \\
\hline
\textbf{Faire} & Nous \textbf{fais}ons & Vous fais\textbf{iez} \\
\hline
\textbf{Venir} & Nous \textbf{ven}ons & Ils ven\textbf{aient} \\
\hline
\end{tabular}
\end{center}

\begin{notebox}
    \begin{remark}
        Il y a une seule exception, le verbe \textit{être}
        \begin{enumerate}
            \item J'étais
            \item Tu étais
            \item Il, elle, on était
            \item Nous étions
            \item Voue étiez
            \item Ils, elles étaient
        \end{enumerate}
    \end{remark}
\end{notebox}

\begin{enumerate}
    \item (\textbf{Modifications de l'orthographe})\footnote{Spelling changes}: Il faut faire attention à l'orthographe de certains verbes:
    \begin{center}
    \begin{tabular}{|c|c|c|}
    \hline
    \textbf{Type de verbe} & \textbf{Présent (nous)} & \textbf{Imparfait} \\
    \hline
    Les verbes en \textbf{-ger} & Nous mang\textbf{e}ons & Nous mang\textbf{ions} / Vous mang\textbf{iez} \\
    \hline
    Les verbes en \textbf{-cer} & Nous pla\textbf{ç}ons & Nous plac\textbf{ions} / Vous plac\textbf{iez} \\
    \hline
    Les verbes en \textbf{-yer} & Nous pa\textbf{y}ons & Nous pay\textbf{ions} / Vous pay\textbf{iez} \\
    \hline
    Les verbes en \textbf{-ier} & Nous étud\textbf{i}ons & Nous étud\textbf{iions} / Vous étud\textbf{iiez} \\
    \hline
    \end{tabular}
    \end{center}
    \item (\textbf{Emploi}): Il y a quatre situations où il faut utiliser l'imparfait.
    \begin{enumerate}
        \item Pour décrire des actions \textbf{habituelles} du passé\footnote{To describe \textbf{habitual} actions of the past}. Par exemple,
        \begin{enumerate}
            \item Quand j'étais petit, je jouais avec mes sœurs.
            \item Mon père nous amenait à la plage pendant l'été.
        \end{enumerate}
        \item Pour décrire le décor d'une action, faire des commentaires, des explications, etc\footnote{the setting of an action, making comments, explanations}. Par exemple,
        \begin{enumerate}
            \item Il était trois heures, il n'y avait personne dans la rue, la pluie tombait\footnote{the rain was falling} et je marchais en silence.
            \item Autrefois\footnote{Formerly} à Paris, il y avait plein de commerçants\footnote{full of merchants} dans la rue.
        \end{enumerate}
        \item Pour exprimer \textbf{l'hypothèse} avec la conjonction \textit{si}. Par exemple,
        \begin{enumerate}
            \item Si j'étais riche, je voyagerais à l'étranger\footnote{travel abroad}.
            \item Elle parle comme si elle avait de l'expérience.
        \end{enumerate}
        \item Pour exprimer une demande atténuée (\textbf{formule de politesse}).\footnote{To express a lesser request (polite formula)} Par exemple,
        \begin{enumerate}
            \item Je voulais vous demander une chose.
            \item Je vous téléphonais pour l'appartement en location.\footnote{the rental apartment}
        \end{enumerate}
    \end{enumerate}
\end{enumerate}

\subsection{Le Conditionnel Présent}
\begin{enumerate}
    \item \textbf{Emploi}
    \begin{itemize}
        \item Pour exprimer une action \textbf{hypothétique}
        \item Pour \textbf{atténuer} une demande, un conseil
    \end{itemize}
    \item (\textbf{Formation du conditionnel présent}): radical du \textbf{futur} + terminaisons de l'\textbf{imparfait} 
    \begin{center}
    \begin{tabular}{>{\bfseries}l c}
    \toprule
    Je \textit{infinitif} & + ais \\
    Tu \textit{infinitif} & + ais \\
    Il/Elle/On \textit{infinitif} & + ait \\
    Nous \textit{infinitif} & + ions \\
    Vous \textit{infinitif} & + iez \\
    Ils/Elles \textit{infinitif} & + aient \\
    \bottomrule
    \end{tabular}
    \end{center}
    \item (\textbf{Exemples de conjugaison})
    \begin{table}[h!]
    \centering
    \begin{tabular}{|c|c|c|c|}
    \hline
    & \textbf{PARLER} & \textbf{FINIR} & \textbf{PRENDRE} \\
    \hline
    je & parler\textbf{ais} & finir\textbf{ais} & prendr\textbf{ais} \\
    tu & parler\textbf{ais} & finir\textbf{ais} & prendr\textbf{ais} \\
    il & parler\textbf{ait} & finir\textbf{ait} & prendr\textbf{ait} \\
    nous & parler\textbf{ions} & finir\textbf{ions} & prend\textbf{}r\textbf{ions} \\
    vous & parler\textbf{iez} & finir\textbf{iez} & prendr\textbf{iez} \\
    ils & parler\textbf{aient} & finir\textbf{aient} & prendr\textbf{aient} \\
    \hline
    \end{tabular}
    \end{table}

    \begin{notebox}
        \begin{remark}
            \textbf{Attention}:
            \begin{enumerate}
                \item Pour les verbes avec ``-re'', Il faut retirer le -e de l'infinitif avant d'ajouter la terminaison.\footnote{The -e must be removed from the infinitive before adding the ending}
                \item Le ``e'' avant le ``r'' n'est généralement pas prononcé et le ``y'' des verbes en \textit{-yer} devient ``i''.\footnote{The ``e'' before the ``r'' is generally not pronounced, and the ``y'' in verbs ending in \textit{-yer} becomes ``i''}
            \end{enumerate}
        \end{remark}
    \end{notebox}
    \item (\textbf{Les irréguliers})
    \begin{table}[h!]
    \centering
    \begin{tabular}{|c|c|c|c|c|c|c|}
    \hline
    \textbf{ALLER} & \textbf{AVOIR} & \textbf{ÊTRE} & \textbf{FAIRE} & \textbf{SAVOIR} & \textbf{POUVOIR} & \textbf{VOULOIR} \\
    \hline
    j'irais & j'aurais & je serais & je ferais & je saurais & je pourrais & je voudrais \\
    \hline
    \end{tabular}
    \end{table}
    
    \begin{table}[h!]
    \centering
    \begin{tabular}{|c|c|c|c|c|c|}
    \hline
    \textbf{VENIR} & \textbf{COURIR} & \textbf{VOIR} & \textbf{DEVOIR} & \textbf{RECEVOIR} & \textbf{ENVOYER} \\
    \hline
    je viendrais & je courrais & je verrais & je devrais & je recevrais & j'enverrais \\
    \hline
    \end{tabular}
    \end{table}
    
    \begin{table}[h!]
    \centering
    \begin{tabular}{|c|c|c|c|c|c|}
    \hline
    \textbf{DEVENIR} & \textbf{MOURIR} & \textbf{FALLOIR} & \textbf{PLEUVOIR} & \textbf{ENVOYER} & \textbf{VALOIR} \\
    \hline
    je deviendrais & je mourrais & il faudrait & il pleuvrait & j'enverrais & je vaudrais \\
    \hline
    \end{tabular}
    \end{table}
\end{enumerate}

\chapter{La Semaine 2}
\section{Leçon 1}
\subsection{Vocabulaire}
\begin{multicols}{2}
    \entry{endormi}{}{adj.}{asleep}{| Je me sens \textbf{endormi}.\footnote{I feel sleepy.}}

    \entry{mauvaise}{}{adj.}{bad}{}

    \entry{habitude}{}{n.f.}{habit}{}

    \entry{arnaque}{}{n.f.}{scam}{}

    \entry{caricature}{}{n.f.}{caricature}{| en chinois, il s'appelle ``man-hua''.}

    \entry{en ligne}{}{}{online}{| faire du shopping \textbf{en ligne}.}

    \entry{partager}{}{v.}{to share}{}

    \entry{jeu}{}{n.m.}{game}{| Il joue beaucoup aux \textbf{jeux} vidéos\footnote{the video games}}

    \entry{privée}{}{adj.}{private}{| la vie \textbf{privée}\footnote{privacy}}

    \entry{réseau}{}{n.m.}{network}{| les \textbf{réseaux sociaux}\footnote{the social networks}}

    \entry{seul}{}{adj.}{lonely}{| Il se sens \textbf{seul}.\footnote{He feels lonely.}}

    \entry{écran}{}{n.m.}{screen}{}

    \entry{recevoir}{}{v.}{to receive}{| Il \textbf{reçoit} tout les temps\footnote{This is just an \textbf{adv} which means ``always''} des notifications}

    \entry{incruster}{}{v.}{to embed}{| Les ecrans sont \textbf{incrustés} dans tous les domaines.}

    \entry{enregistrer}{}{v.}{to record}{| Les réseaux sociaux \textbf{enregistre} tes centres d'intérêt.\footnote{Social networks record your interests.}}

    \entry{gaspillage}{}{n.m.}{waste}{}

    \entry{rendre}{}{v.}{to make}{| Les vidéos qui nous \textbf{rendent} mal à l'aise\footnote{The videos that make us uncomfortable.}}

    \entry{se détendre}{}{v.}{to relax oneself}{| Il offre une autre méthode pour \textbf{se détendre}.}
\end{multicols}

\subsection{Phrase}
\begin{enumerate}
    \item On ne peut pas \textbf{rencontrer dans la vraie vie}.\footnote{We cannot meet in the real life.}

    \item C'est important \textbf{de ne pas} parler, \textbf{de ne pas} partager/poster des informations personnelles sur internet.\footnote{It is important not to talk, share/post personal information on the internet}

    \item \textbf{L'addiction aux} jeux vidéo.\footnote{Video games addiction.}

    \item \textbf{être accros aux} quelque chose = \textbf{avoir une addiction pour} quelque chose.

    \item C'est \textbf{chouette}\footnote{It's cool.}

    \item être à l'aise\footnote{to be comfortable} $\neq$ être mal à l'aise.
\end{enumerate}

\section{Leçon 2}
\subsection{Vocabulaire}
\begin{multicols}{2}
    \entry{déchet}{}{n.m.}{rubbish}{}

    \entry{étonnant}{}{adj.}{surprising}{}

    \entry{cerveau}{}{n.m.}{brain}{}

    \entry{pertinence}{}{n.f.}{high relevance}{}

    \entry{panacée}{}{n.f.}{panacea}{| Solution miracle à un problème / Remède universel}

    \entry{plutôt}{}{adv.}{rather}{| Je choisirais \textbf{plutôt} celui-ci que celui-la.}

    \entry{sanitaire}{}{adj.}{related to health}{| relatif à la santé}

    \entry{désastre}{}{n.m.}{disaster}{}

    \entry{refuge}{}{n.m.}{shelter}{}

    \entry{reprocher}{}{v.}{to scold}{}

    \entry{évoquer}{}{v.}{to mention}{| le même que \textbf{mentionner}}

    \entry{nuisible}{}{adj.}{harmful}{| Les écrans sont \textbf{nuisibles} pour les enfants.}

    \entry{espace}{}{n.m.}{space}{}

    \entry{un pave tactile}{}{n.m.}{touchpad}{}

    \entry{sauvegarder}{}{v.}{to safeguard}{}
\end{multicols}

\subsection{Phrase}
\begin{enumerate}
    \item avis = point de vue\footnote{point of view} = opinion

    \item sous prétexte de\footnote{under the pretext of = excuse}

    \item Pour exprimer la difficulté de faire quelque chose
    \begin{itemize}
        \item C'est difficile de + \textit{verbe à l'infinitif}
        \item J'ai beaucoup de mal avec + quelque chose
        \item Je galère avec\footnote{I am struggling with} + quelque chose
        \item Je ne suis pas très a l'aise avec + quelque chose
    \end{itemize}

    \item Pour exprimer le désir\footnote{desire} de faire quelque chose
    \begin{itemize}
        \item J'aimerais savoir + \textit{verbe à l'infinitif}
    \end{itemize}
\end{enumerate}

\section{Leçon 3}
\subsection{Vocabulaire}
\begin{multicols}{2}
    \entry{carton}{}{n.m.}{cardboard}{| la maison en \textbf{carton}}

    \entry{écologique}{}{adj.}{ecological}{| Relatif à l'écologie.}

    \entry{recyclé}{}{adj.}{recycled}{| les cartons \textbf{recyclés}}

    \entry{recyclable}{}{adj.}{recyclable}{}

    \entry{durée de vie}{}{n.f.}{lifespan}{}

    \entry{éco-responsable}{}{adj.}{eco-responsible}{}

    \entry{l'énergie solaire}{}{n.f.}{solar energy}{}

    \entry{un système de ventilation}{}{n.m.}{ventilation system}{}

    \entry{monter}{}{v.}{to assemble}{| monter un meuble\footnote{to assemble a piece of furniture}}

    \entry{démonter}{}{v.}{to disassemble}{}

    \entry{forêt}{}{n.f.}{forest}{}

    \entry{vivant}{}{adj.}{alive}{| C'est une maison \textbf{vivante}}

    \entry{béton}{}{n.m.}{concrete}{}

    \entry{mer}{}{n.f.}{sea}{}

    \entry{hauteur}{}{n.f.}{height}{| Il est 175 de \textbf{hauteur}}

    \entry{magnifique}{}{adj.}{magnificent}{| wonderful}
\end{multicols}

\subsection{Phrase}
\begin{enumerate}
    \item \textbf{C'est construit à partir de}\footnote{It is built from} + quelque chose \\
    e.g., C'\textbf{est construit à partir de} plusieurs couches\footnote{layers} de cartons
    \item \textbf{à base de} + quelque chose\footnote{be based on something} \\
    e.g., C'est une maison écologique construite \textbf{à base de} carton recyclé.
    \item quelque chose être \textbf{à la pointe de la technologie}\footnote{at the cutting edge of technology}
    \item \textbf{être résistante a} + quelque chose\footnote{be resistant to} \\
    e.g., Elle \textbf{est résistante aux} climats difficiles.
    \item \textbf{Elle se trouve où?}\footnote{Where is it located?}
    \item \textbf{être suitée} + prep. + un lieu\footnote{be located at} \\
    e.g., Elle \textbf{est située} dans la forêt.
\end{enumerate}

\section{Leçon 4}
\subsection{Vocabulaire}
\begin{multicols}{2}
    \entry{les outils numériques}{}{}{digital tools}{}

    \entry{surtout}{}{adv.}{especially}{}

    \entry{bouger}{}{v.}{to move}{}

    \entry{néfaste}{}{adj.}{harmful}{| les consequences \textbf{néfastes}}

    \entry{tranquille}{}{adj.}{quiet}{}
\end{multicols}

\subsection{Phrase}
\begin{enumerate}
    \item à notre époque moderne\footnote{in our modern era}
    \item \textbf{des problèmes de}\footnote{the problem of} + quelque chose \\
    e.g., \textbf{des problèmes de} sociabilité, de langage, etc.
    \item \textbf{manque de}\footnote{the lack of} + quelque chose \\
    e.g., \textbf{manque de} vocabulaire
\end{enumerate}

\section{Leçon 5}
\subsection{Vocabulaire}
\begin{multicols}{2}
    \entry{rêve}{}{n.m.}{dream}{}

    \entry{siège}{}{n.m.}{seat}{}

    \entry{stimuler}{}{v.}{to stimulate}{| \textbf{stimuler} l'imagination}
\end{multicols}

\subsection{Phrase}
\begin{enumerate}
    \item \textbf{le plutôt possible}\footnote{as soon as possible}

    \item \textbf{C'est prouvé par}\footnote{It is proved by} + quelque chose \\
    e.g., \textbf{C'est prouvé par} des études scientifiques
\end{enumerate}

\section{Grammaire}
\subsection{Le Subjonctif Présent}
En français, on utilise plusieurs structures impersonnelles avec \textit{il} suivies par le subjonctif.\footnote{Le \textit{il} ne représente personne et ne peut pas être changé par un autre pronom.} Par exemple, Il faut que, Il vaut mieux que\footnote{It is better that}, etc. Tout s'ajoute\footnote{adds itself with} \textbf{sujet + subjonctif}

\begin{notebox}
    \begin{remark}
    \textbf{Attention}:
    \begin{enumerate}
        \item Les structures avec \textit{``il est''} expriment une opinion \textbf{plus personnelle}, \textit{``c'est''} est utilisé pour \textbf{la généralité}.
        \item Après une construction impersonnelle, il n'y a pas toujours le subjonctif. Il est aussi possible d'avoir l'indicatif.
        \begin{itemize}
            \item \textbf{Le subjonctif} est utilisé pour parler \textbf{d'une réalité incertaine}.
            \item \textbf{L'indicatif} pour \textbf{une réalité certaine} ou probable.
        \end{itemize}
        Et tous les cas sont les suivants,
        
        \centerline{\includegraphics[width=0.5\linewidth]{images/1.png}}
    \end{enumerate}
    \end{remark}
\end{notebox}

\begin{enumerate}
    \item \textbf{Emploi}: Pour exprimer une action \textbf{éventuelle}, dont la réalisation n'est \textbf{pas certaine}.
    \item (\textbf{Conjugaison Règle}): Radicale de la $3^{\text{ème}}$ personne du pluriel\footnote{ils/elles} du présent de l'indicatif. Retirez le \textit{-ent} et suivez la règle suivante.
    \begin{table}[h]
    \centering
    \begin{tabular}{|>{\centering\arraybackslash}p{2.5cm}|>{\centering\arraybackslash}p{2.5cm}|>{\centering\arraybackslash}p{2.5cm}|>{\centering\arraybackslash}p{2.5cm}|}
    \hline
    \textbf{\textcolor{red}{QUE} + Personne} & \textbf{1er groupe radical +} & \textbf{2ème groupe radical +} & \textbf{3ème groupe radical +} \\
    \hline
    Que je/j' & \textcolor{red}{\textbf{e}} & \textcolor{red}{\textbf{isse}} & \textcolor{red}{\textbf{e}} \\
    \hline
    Que tu & \textcolor{red}{\textbf{es}} & \textcolor{red}{\textbf{isses}} & \textcolor{red}{\textbf{es}} \\
    \hline
    Qu'il/elle/on & \textcolor{red}{\textbf{e}} & \textcolor{red}{\textbf{isse}} & \textcolor{red}{\textbf{e}} \\
    \hline
    Que nous & \textcolor{red}{\textbf{ions}} & \textcolor{red}{\textbf{issions}} & \textcolor{red}{\textbf{ions}} \\
    \hline
    Que vous & \textcolor{red}{\textbf{iez}} & \textcolor{red}{\textbf{issiez}} & \textcolor{red}{\textbf{iez}} \\
    \hline
    Qu'ils/elles & \textcolor{red}{\textbf{ent}} & \textcolor{red}{\textbf{issent}} & \textcolor{red}{\textbf{ent}} \\
    \hline
    \end{tabular}
    \caption{Les terminaisons du subjonctif présent}
    \end{table}

    \begin{notebox}
        \begin{remark}
            Le 1er groupe est les verbes avec \textbf{-er}, le 2eme groupe est les verbes avec \textbf{-ir}, le 3eme groupe est les verbes avec \textbf{-re}.
        \end{remark}
    \end{notebox}
    \item (Des exemples)
    \begin{table}[h]
    \centering
    \begin{tabular}{|l|l|}
        \hline
        \textbf{Indicatif (3e personne)} & \textbf{Subjonctif (1e personne)} \\
        \hline
        Ils part\sout{ent} & que je part\textbf{e} \\
        \hline
        Ils mett\sout{ent} & que je mett\textbf{e} \\
        \hline
        Ils lis\sout{ent} & que je lis\textbf{e} \\
        \hline
        Ils écriv\sout{ent} & que j'écriv\textbf{e} \\
        \hline
        \end{tabular}
    \end{table}
    \item (\textbf{Les irréguliers})
    \begin{center}
    \begin{longtable}{|c|c|c|c|c|}
    \caption{Verbes irréguliers au subjonctif} \\
    \hline
    \multicolumn{5}{|c|}{\textbf{\textcolor{blue}{Verbes irréguliers}}} \\
    \hline
    & \textbf{avoir} & \textbf{être} & \textbf{aller} & \textbf{faire} \\
    \hline
    \multirow{6}{*}{\textbf{que/qu'}} 
    & j'aie & je sois & j'aille & je fasse \\
    \cline{2-5}
    & tu aies & tu sois & tu ailles & tu fasses \\
    \cline{2-5}
    & il ait & il soit & il aille & il fasse \\
    \cline{2-5}
    & nous ayons & nous soyons & nous allions & nous fassions \\
    \cline{2-5}
    & vous ayez & vous soyez & vous alliez & vous fassiez \\
    \cline{2-5}
    & ils aient & ils soient & ils aillent & ils fassent \\
    \hline
    & \textbf{pouvoir} & \textbf{valoir} & \textbf{savoir} & \textbf{vouloir} \\
    \hline
    \multirow{6}{*}{\textbf{que/qu'}} 
    & je puisse & je vaille & je sache & je veuille \\
    \cline{2-5}
    & tu puisses & tu vailles & tu saches & tu veuilles \\
    \cline{2-5}
    & il puisse & il vaille & il sache & il veuille \\
    \cline{2-5}
    & nous puissions & nous valions & nous sachions & nous voulions \\
    \cline{2-5}
    & vous puissiez & vous valiez & vous sachiez & vous vouliez \\
    \cline{2-5}
    & ils puissent & ils vaillent & ils sachent & ils veuillent \\
    \hline
    \end{longtable}
    \end{center}
    \vspace{-30pt} % or -5pt or -8mm etc.
\end{enumerate}

\subsection{Le Participe Présent}
Le participe présent est utilisé à l'écrit pour remplacer \textit{qui + verbe}.

\begin{enumerate}
    \item (\textbf{Conjugaison Règle}): Trouvez la première personne du pluriel (\textit{nous}) du présent de l'indicatif. Retirez le \textit{-ons} et ajoutez\footnote{add} le \textit{-ant}. Par exemple,
    \begin{table}[h]
    \centering
    \begin{tabular}{|l|l|l|}
    \hline
    \textbf{Infinitif} & \textbf{1ère personne pluriel} & \textbf{Participe présent} \\
    \hline
    Parler & nous \textbf{parlons} & \textbf{parlant} \\
    \hline
    Finir & nous \textbf{finissons} & \textbf{finissant} \\
    \hline
    Prendre & nous \textbf{prenons} & \textbf{prenant} \\
    \hline
    Aller & nous \textbf{allons} & \textbf{allant} \\
    \hline
    \end{tabular}
    \caption{Formation du participe présent}
    \end{table}
\end{enumerate}

\subsection{La Négation}
\begin{enumerate}
    \item (\textbf{Règle})
    \begin{center}
    \begin{longtable}{|p{3.5cm}|p{6cm}|}
    \hline
    \textbf{Forme affirmative} & \textbf{Forme négative} \\
    \hline
    \endfirsthead
    
    \hline
    \textbf{Forme affirmative} & \textbf{Forme négative} \\
    \hline
    \endhead
    
    \hline
    \endfoot
    
    \hline
    \caption{Les formes de négation en français} \\
    \endlastfoot
    
    Encore & ne.........plus \\
    \hline
    Déjà & ne..........pas encore \\
    \hline
    Toujours, souvent & ne.......jamais \\
    \hline
    Quelqu'un, tout le monde & ne........personne / personne ne...... \\
    \hline
    Tout, quelque chose & ne.......rien / rien ne...... \\
    \hline
    Et / ou & ne....... ni....... ni...... / Ni....... ni.........ne..... \\
    \hline
    \end{longtable}
    \end{center}
    \vspace{-60pt} % or -5pt or -8mm etc.
    \begin{notebox}
        \begin{remark}
            Le verbe est toujours \textbf{après} \textit{ne}.
        \end{remark}
    \end{notebox}
\end{enumerate}

\subsection{Le Futur Simple}
Le futur simple exprime un fait ou une action qui se déroulera plus tard, elle n'a pas encore eu lieu au moment où nous nous exprimons.\footnote{The simple future expresses a fact or action that will take place later; it has not yet taken place at the time we are speaking.}

\begin{enumerate}
    \item (\textbf{Emploi})
    \begin{itemize}
        \item Lorsqu'on situe un événement ou un fait dans un avenir assez, voire très lointain (dans une semaine, l'année prochaine, etc.)\footnote{when we place an event or fact in a fairly, or even very, distant future (in a week, next year, etc.)} \\
        e.g., J'habiterai encore à Paris dans deux ans
        \item Dans les phrases qui contiennent une proposition subordonnée de temps introduite par \textbf{quand, lorsque, dès que}.\footnote{In sentences containing a subordinate clause of time introduced by quand, when, as soon as.}\\
        e.g., Nous sortirons quand la pluie s'arrêtera.
    \end{itemize}
    \item (\textbf{Règles Conjugaison})
    \begin{center}
        \begin{tabular}{|p{4cm}|p{4cm}|p{4.5cm}|}
        \hline
        \centering Groupe & Formation & Exemple (je) \\
        \hline
        1\textsuperscript{er} groupe \newline (verbes en -ER) & Infinitif \textbf{complet} + terminaisons & \textit{manger} $\rightarrow$ je \textbf{mangerai} \\
        \hline
        2\textsuperscript{e} groupe \newline (verbes en -IR) & Infinitif \textbf{complet} + terminaisons & \textit{finir} $\rightarrow$ je \textbf{finirai} \\
        \hline
        3\textsuperscript{e} groupe \newline (verbes en -RE) & Infinitif \textbf{sans -E} + terminaisons & \textit{attendre} $\rightarrow$ j'\textbf{attendrai} \\
        \hline
        \end{tabular}
    \end{center}
    \item (\textbf{Les terminaisons du futur simple})
    \begin{center}
        \begin{tabular}{|c|c|c|c|c|l|}
        \hline
        je  & tu  & il/elle/on & nous & vous & ils/elles \\ \hline
        \textbf{-ai} & \textbf{-as} & \textbf{-a}         & \textbf{-ons} & \textbf{-ez}  & \textbf{-ont}      \\ \hline
        \end{tabular}
    \end{center}
    \begin{notebox}
        \begin{remark}
            \textbf{Attention}:
            \begin{enumerate}
                \item Les verbes en \textit{``eler''} ou \textit{``eter''} doublent leur \textbf{consonne} ou \textbf{prennent un accent}.
                    \begin{center}
                        appe\textbf{l}er $\Rightarrow$ j'appe\textbf{ll}erai \\
                        acheter $\Rightarrow$ j'ach\textbf{è}terai
                    \end{center}
                \item Les verbes en \textit{``yer''} changent le \textit{y} en \textit{i}.
                    \begin{center}
                        essu\textbf{y}er $\Rightarrow$ j'essu\textbf{i}erai
                    \end{center}
            \end{enumerate}
        \end{remark}
    \end{notebox}
    \item \textbf{La différence entre Future Simple et Future Proche}
    \begin{longtable}{|>{\centering\arraybackslash}m{5.5cm}|>{\centering\arraybackslash}m{5.5cm}|}
    \hline
    \textbf{Futur simple} & \textbf{Futur proche} \\
    \hline
    \endfirsthead
    
    \hline
    \textbf{Futur simple} & \textbf{Futur proche} \\
    \hline
    \endhead
    
    \hline
    \endfoot
    
    \hline
    \endlastfoot
    
    Plus formel & Plus familier \\
    \hline
    Utilisé à l'écrit & Utilisé à l'oral \\
    \hline
    \end{longtable}

    \item (\textbf{Les irréguliers})
    \begin{table}[h]
    \centering
    \begin{tabular}{|l|l|l|l|}
    \hline
    \textbf{Infinitif} & \textbf{1ère personne} & \textbf{Infinitif} & \textbf{1ère personne} \\
    \hline
    être & je serai & savoir & je saurai \\
    \hline
    avoir & j'aurai & tenir & je tiendrai \\
    \hline
    aller & j'irai & venir & je viendrai \\
    \hline
    devoir & je devrai & voir & je verrai \\
    \hline
    envoyer & j'enverrai & vouloir & je voudrai \\
    \hline
    faire & je ferai & valoir & je vaudrai \\
    \hline
    pouvoir & je pourrai & pleuvoir & il pleuvra \\
    \hline
    recevoir & je recevrai & falloir & il faudra \\
    \hline
    \end{tabular}
    \caption{Verbes irréguliers au futur simple}
    \label{tab:irregular_verbs}
    \end{table}
\end{enumerate}

\subsection{Le Pronom Possessif}
Le pronom possessif marque l'appartenance, la possession. Comme les autres pronoms, il remplace un nom ou un groupe nominal.\footnote{The possessive pronoun indicates belonging, possession. Like other pronouns, it replaces a noun or a noun phrase.} e.g., en anglais, it's like \textit{mine, yours, his, hers, etc}.

\begin{enumerate}
    \item (\textbf{Règle}): Le pronom possessif est formé d'un article défini (\textit{le, la} ou \textit{les}) et du mot \textit{mien, tien, sien, nôtre, vôtre} ou \textit{leur}. Le pronom s'accorde \textbf{en genre} et \textbf{en nombre} avec son antécédent. 

    \begin{center}
        \includegraphics[width=1\linewidth]{images/2.png}
    \end{center}
    \item (\textbf{Exemples}): J'ai acheté mes skis en Italie. Pierre a acheté \textbf{les siens} en France.
\end{enumerate}

\chapter{La Semaine 3}
\section{Leçon 1}
\subsection{Vocabulaire}
\begin{multicols}{2}
    \entry{ailleurs}{}{adv.}{elsewhere}{| Dans un autre lieu}

    \entry{vivre}{}{v.}{to live}{| 10 habitants \textbf{vivent} en Italie.}

    \entry{compter}{}{v.}{to count}{| La Slovénie \textbf{compte} 2 habitants. $\equiv$ En Slovénie, il y a 2 habitants.}

    \entry{légèrement}{}{adv.}{slightly}{}

    \entry{manière}{}{n.f.}{manner}{}

    \entry{baisse}{}{n.f.}{drop}{| la \textbf{baisse} de natalité\footnote{The drop of birth rate}}

    \entry{hausse}{}{n.f.}{increase}{| la \textbf{hausse} de mortalité\footnote{The increase of death rate}}

    \entry{tranche}{}{n.f.}{slice}{}

    \entry{ampleur}{}{n.f.}{magnitude}{}

    \entry{espérance}{}{n.f.}{hope}{| L'espérance de vie\footnote{Life expectancy}}

    \entry{écart}{}{n.m.}{gap}{}
\end{multicols}

\section{Leçon 2}
\subsection{Vocabulaire}
\begin{multicols}{2}
    \entry{la colline}{}{n.f.}{hill}{}

    \entry{la bosquet}{}{n.f.}{grove}{| en chinois, c'est ``树林''}

    \entry{la marais}{}{n.f.}{marshe}{| en chinois, c'est ``沼泽''}

    \entry{la forêt}{}{n.f.}{forest}{}

    \entry{le sommet}{}{n.m.}{summit}{}

    \entry{le plateau}{}{n.m.}{plateau}{| en chinois, c'est ``高原''}

    \entry{le col}{}{n.m.}{col}{| en chinois, c'est ``山口''}

    \entry{le gorge}{}{n.m.}{gorge}{| en chinois, c'est ``峡谷''}

    \entry{la plaine}{}{n.f.}{plain}{| en chinois, c'est ``平原''}

    \entry{le massif}{}{n.m.}{massif}{| en chinois, c'est ``山丘''}

    \entry{le cap}{}{n.m.}{cap}{| en chinois, c'est ``海角''}

    \entry{le golfe}{}{n.m.}{golf}{| en choinois, c'est ``海湾''}

    \entry{la falaise}{}{n.f.}{cliff}{}

    \entry{le affluent}{}{n.m.}{tributary}{| en choinois, c'est ``支流''}

    \entry{le biome}{}{n.m.}{biome}{| en chinois, c'est ``生物群落区''}

    \entry{d'outre-mer}{}{adj.}{overseas}{}

    \entry{siècle}{}{n.m.}{century}{}

    \entry{aventurier}{}{n.m.}{adventurer}{}

    \entry{itinéraire}{}{n.m.}{route}{}

    \entry{gravir}{}{v.}{to climb}{}

    \entry{ensoleillé}{}{adj.}{sunny}{}

    \entry{île}{}{n.f.}{island}{}

    \entry{posséder}{}{v.}{to have}{| la même chose que ``avoir''}

    \entry{relief}{}{n.m.}{relief}{| en chinois, c'est ``地貌''}

    \entry{principalement}{}{adv.}{mainly}{}

    \entry{actif}{}{adj.}{active}{}
\end{multicols}

\subsection{Phrase}
\begin{enumerate}
    \item \textbf{au sud / au nord / à l'est / à l'ouest} de\footnote{at the south/north/east/west of} + quelque chose
\end{enumerate}

\section{Leçon 3}
\subsection{Vocabulaire}
\begin{multicols}{2}
    \entry{le glissement de terrain}{}{n.m.}{landside}{| en choinois, c'est ``滑坡''}

    \entry{l'éruption volcanique}{}{n.f.}{volcanic eruption}{| en chinois, c'est ``火山喷发''}

    \entry{inondation}{}{n.f.}{flood}{}

    \entry{le tsunami}{}{n.m.}{tsunami}{}

    \entry{le séisme}{}{n.m.}{earthquake}{}

    \entry{l'incendie}{}{n.m.}{fire}{}

    \entry{la tornade}{}{n.f.}{tornado}{}

    \entry{la canicule}{}{n.f.}{heatwave}{}

    \entry{l'onde}{}{n.f.}{wave}{}

    \entry{la vague}{}{n.f.}{(sea) wave}{}

    \entry{le fond}{}{n.m.}{bottom}{}

    \entry{le dégât}{}{n.m.}{damage}{}

    \entry{le recul}{}{n.m.}{moving back}{}

    \entry{la dizaine}{}{n.f.}{ten}{}

    \entry{le mascaret}{}{n.m.}{tidal bore}{}

    \entry{propager}{}{v.}{to spread}{}

    \entry{la secousse}{}{n.f.}{tremor}{| en chinois, c'est ``发抖''}

    \entry{ralentir}{}{v.}{to slow down}{}

    \entry{retenir}{}{v.}{to catch hold of sth}{}

    \entry{côtier}{}{adj.}{coastal}{}

    \entry{lorsque}{}{conj.}{when}{| la même chose que ``quand''}
\end{multicols}

\subsection{Phrase}
\begin{enumerate}
    \item \textbf{contrairement à}\footnote{contrary to} + quelque chose

    \item \textbf{la semaine suivante}\footnote{The following week}
\end{enumerate}

\section{Leçon 4}
\subsection{Vocabulaire}
\begin{multicols}{2}
    \entry{cosmétique}{}{n.m.}{beauty product}{}

    \entry{agroalimentaire}{}{adj.}{food-processing}{}

    \entry{édition}{}{n.f.}{publication}{}

    \entry{touristique}{}{adj.}{touristic}{}

    \entry{mondial}{}{adj.}{global}{}

    \entry{mondialisation}{}{n.f.}{globalization}{}

    \entry{détrôner}{}{v.}{to dethrone}{}

    \entry{descendre}{}{v.}{to go down}{}

    \entry{augmenter}{}{v.}{to increase}{}

    \entry{diriger}{}{v.}{to guide/lead}{}
\end{multicols}

\subsection{Phrase}
\begin{enumerate}
    \item \textbf{à l'inverse,}\footnote{Conversely,} + quelque phrase
\end{enumerate}



\section{Leçon 5}
\subsection{Vocabulaire}
\begin{multicols}{2}
    \entry{respectivement}{}{adv.}{respectively}{}

    \entry{peuple}{}{n.m.}{people}{}

    \entry{pouvoir}{}{n.m.}{power}{}

    \entry{démocratie}{}{n.f.}{democracy}{}

    \entry{élire}{}{v.}{to elect}{}
\end{multicols}

\section{Grammaire}
\subsection{Les Pronoms Relatifs}
Les \textbf{pronoms relatifs} sont utilises pour remplacer un nom et réunir deux phrases ou mettre en relief. On le choisit selon la fonction du mot remplacé.\footnote{Relative pronouns are used to replace a noun and join two sentences or to emphasize them. They are chosen according to the function of the replaced word.}
\subsubsection{Les Pronoms Relatifs Simples}
\begin{enumerate}
    \item \textbf{qui}: Reprend le \textbf{sujet} du verbe qui suit\footnote{Takes the \textbf{subject} of the following verb}:
    \begin{center}
        \textcolor{orange}{qui + verbe}
    \end{center}
    e.g., La fille \textbf{qui} parle st ma sœur.
    \begin{notebox}
        \begin{remark}
            Après une préposition, remplace une personne (jamais un objet) \\
            e.g., Voici la femme \underline{chez} \textbf{qui} j'habite.
        \end{remark}
    \end{notebox}
    \item \textbf{que}: Reprend le \textbf{complément d'objet} (COD) du verbe qui suit:
    \begin{center}
        \textcolor{orange}{que + sujet + verbe}
    \end{center}
    e.g., Les documents \textbf{que} j'ai mis sur la table sont pour toi.
    \item \textbf{où}: Reprend \textbf{le lieu ou le moment}:
    \begin{center}
        \textcolor{orange}{où + sujet + verbe}
    \end{center}
    e.g., Regarde! C'est l'entreprise \textbf{où} je travaille.
    \item \textbf{dont}: Reprend le \textbf{complément} + \textbf{de}
    \begin{center}
        \textcolor{orange}{dont + sujet + verbe}
    \end{center}
    e.g., La seule chose \textbf{dont} \underline{j'ai besoin} c'est dormir. \\
    Voici le tableau des verbes et expressions suivis de “de”:
    \begin{center}
    \begin{longtable}{|>{\centering\arraybackslash}m{6cm}|>{\centering\arraybackslash}m{6cm}|}
    \hline
    \textbf{Français} & \textbf{English} \\
    \hline
    \endfirsthead
    
    \hline
    \textbf{Français} & \textbf{English} \\
    \hline
    \endhead
    
    \hline
    \endfoot
    
    \hline
    \caption{Verbes et expressions suivis de ``de''} \\
    \endlastfoot
    
    avoir besoin de & to need \\
    \hline
    avoir peur de & to be afraid of \\
    \hline
    avoir envie de & to want / feel like \\
    \hline
    avoir l'habitude de & to be used to \\
    \hline
    parler de & to talk about \\
    \hline
    être responsable de & to be responsible for \\
    \hline
    être conscient de & to be aware of \\
    \hline
    être fan de & to be a fan of \\
    \hline
    être sûr / certain de & to be sure / certain of \\
    \hline
    se servir de & to make use of \\
    \hline
    s'occuper de & to take care of \\
    \hline
    abuser de & to abuse / misuse \\
    \hline
    s'agir de & to be about \\
    \hline
    s'assurer de & to make sure of \\
    \hline
    avoir honte de & to be ashamed of \\
    \hline
    avoir l'intention de & to intend to \\
    \hline
    avoir raison / tort de & to be right / wrong to \\
    \hline
    discuter de & to discuss \\
    \hline
    douter de & to doubt \\
    \hline
    s'éloigner de & to move away from \\
    \hline
    être amoureux de & to be in love with \\
    \hline
    être désolé de & to be sorry about \\
    \hline
    jouer de (musique) & to play (a musical instrument) \\
    \hline
    se moquer de & to make fun of \\
    \hline
    rêver de & to dream of / want \\
    \hline
    venir de (passé récent) & to have just (done something) \\
    \hline
    la manière / façon de & the way of \\
    \end{longtable}
    \end{center}
    \vspace{-60pt} % or -5pt or -8mm etc.
\end{enumerate}

\subsubsection{Les Pronoms Relatifs Composés}
Le prénom relatif composé est formé d'une \textbf{préposition + lequel / laquelle / lesquels / lesquelles} selon le \textbf{genre} et le \textbf{nombre} du nom qu'il remplace. 
\begin{enumerate}
    \item \textbf{préposition + lequel / laquelle / lesquels / lesquelles}: \\
    e.g., Il habit \textbf{dans} \underline{cet appartement.} $\Rightarrow$ C'est l'appartement \textbf{dans lequel} il habite. \\
    e.g., J'achète ces lunettes. J'ai économisé \textbf{pour} \underline{ces lunettes.} $\Rightarrow$ J'achète ces lunettes \textbf{pour lesquelles} j'ai économisé.
    \item \textbf{auquel / à laquelle / auxquels / auxquelles}: \textbf{Si la préposition est ``à''}, certaines contractions sont nécessaires, \textbf{auquel}(à + lequel), \textbf{à laquelle}, \textbf{auxquels}(à + lesquels), \textbf{auxquelles}(à + lesquelles).\\
    e.g., J'adore ce film. Je pense souvent \textbf{à} \underline{ce film.} $\Rightarrow$  J'adore ce film \textbf{auquel} je pense souvent.
    \item \textbf{duquel / de laquelle / desquels / desquelles}: \textbf{Si la préposition est ``de''}, certaines contractions sont nécessaires, \textbf{duquel}(de + lequel), \textbf{de laquelle}, \textbf{desquels}(de + lesquels), \textbf{desquelles}(de + lesquelles).\\
    e.g., Il vit dans cette maison. \textbf{À côté de} \underline{cette maison}, il y a une église $\Rightarrow$ Il vit dans cette maison \textbf{à côté de laquelle} il y a une église.
    \item \textbf{préposition + qui}: Si on remplace \textbf{une personne}, on peut utiliser \textbf{qui} au lieu de\footnote{instead of} \textbf{lequel, laquelle, lesquels, lesquelles}. \\
    e.g., Je m'entends bien \textbf{avec} \underline{ces collègues.} $\Rightarrow$ Ce sont des collègues \textbf{avec lesquels / avec qui} je m'entends bien.
\end{enumerate}

\subsection{La Cause et La Conséquence}
\subsubsection{La Cause}
\begin{enumerate}
    \item \textbf{Parce que}: introduit une cause et \textbf{répond à ``pourquoi''} \\
    e.g. Le match a été ennuie, \textbf{parce qu}'il pleuvait.
    \item \textbf{Comme}: place la case \textbf{en début} de phrase \\
    e.g., \textbf{Comme} il pleuvait, le match a été annulé.
    \item \textbf{Car}: c'est ``parce que'' en langage formel ou à l'écrit \\
    e.g., Le match a été annulé, \textbf{car} il pleuvait.
    \item \textbf{Puisque}: introduit une cause connue ou évidente \\
    e.g., \textbf{Puisque} le match est annulé, ils doivent rembourser.
    \item \textbf{À cause de}: cause \textbf{négative ou neutre} \\
    e.g., J'ai raté \textbf{à cause de} toi.
    \item \textbf{Grâce à}: cause \textbf{positive} \\
    e.g., J'ai réussi \textbf{grâce à} toi.
\end{enumerate}
\subsubsection{La Conséquence}
\begin{enumerate}
    \item \textbf{Donc}: exprime une conséquence qui s'apparente à une évidence \\
    e.g., Je n'ai pas 18 ans \textbf{donc} je ne peux conduire une voiture.
    \item \textbf{C'est pourquoi}: exprime également une conséquence mais \textbf{l'explique}: \\
    e.g., Je ne suis pas capable de traiter\footnote{treat} ce problème, \textbf{c'est pourquoi} j'ai fait appel à un spécialiste.
    \item \textbf{C'est la raison pour laquelle}: s'utilise plutôt à l'écrit ou dans un langage plus formel: \\
    e.g., Je désire apprendre le français. \textbf{C'est la raison pour laquelle} je souhaiterais m'inscrire\footnote{record} dans votre cours.
    \item \textbf{Alors} et \textbf{c'est pour ça que}: s'utilisent quant à\footnote{as to} eux en langage courant\footnote{common}: \\
    e.g., J'ai une réunion à deux heures \textbf{alors} je vais partir à une heure.
\end{enumerate}

\chapter{Le Semaine 4}
\section{Leçon 1}
\subsection{Vocabulaire}
\begin{multicols}{2}
    \entry{drôle}{}{adj.}{funny}{| la même chose que ``amusante''}

    \entry{supposition}{}{n.f.}{supposition}{| la meme chose que ``hypothèse''}

    \entry{fragile}{}{adj.}{fragile}{}

    \entry{fiable}{}{adj.}{reliable}{}

    \entry{contradictoire}{}{adj.}{contradictory}{}
\end{multicols}

\subsection{Phrase}
\begin{enumerate}
    \item \textbf{croire à}\footnote{believe in} + quelque chose
    \item \textbf{je suis du même avis que}\footnote{I agree with} + quelqu'on
\end{enumerate}

\section{Leçon 2}
\subsection{Vocabulaire}
\begin{multicols}{2}
    \entry{pronostic}{}{n.m.}{prediction}{}

    \entry{équipe}{}{n.f.}{team}{}

    \entry{gagner}{}{v.}{to win/earn}{}
\end{multicols}

\subsection{Phrase}
\begin{enumerate}
    \item \textbf{Je crois que}\footnote{I believe that} + quelque chose
    \item \textbf{C'est certain que} + quelque chose
\end{enumerate}

\section{Lecon 3}
\subsection{Vocabulaire}
\subsubsection*{L'hébergement}
{\small accommodation}
\begin{multicols}{2}
    \entry{le camping}{}{n.m.}{camping}{}

    \entry{l'auberge de jeunesse}{}{n.f.}{cheap hotel}{}

    \entry{la pension}{}{n.f.}{boarding school}{}

    \entry{la location de vacances}{}{n.f.}{vacation rental}{}

    \entry{la chambre d'hôtes}{}{n.f.}{guest room}{}
\end{multicols}

\subsubsection*{Les moyens de transport}
{\small means of transport}
\begin{multicols}{2}
    \entry{la voiture}{}{n.f.}{car}{}
    
    \entry{le train}{}{n.m.}{train}{}
    
    \entry{l'avion}{}{n.m.}{airplane}{}
    
    \entry{le bateau}{}{n.m.}{boat}{}
    
    \entry{le bus / le car}{}{n.m.}{bus / coach}{}
    
    \entry{le taxi}{}{n.m.}{taxi}{}
    
    \entry{le camping-car}{}{n.m.}{motorhome}{}
    
    \entry{le mobile home}{}{n.m.}{mobile home}{}
\end{multicols}

\subsubsection*{La réservation}
{\small booking}
\begin{multicols}{2}
    \entry{l'agence de voyages}{}{n.f.}{travel agency}{}
    
    \entry{la gare}{}{n.f.}{train station}{}
    
    \entry{l'aéroport}{}{n.m.}{airport}{}
    
    \entry{le guichet}{}{n.m.}{ticket counter}{}
    
    \entry{les horaires}{}{n.m. pl.}{schedules}{}
    
    \entry{le billet}{}{n.m.}{ticket}{}
    
    \entry{l'aller simple}{}{n.m.}{one-way trip}{}
    
    \entry{l'aller-retour}{}{n.m.}{round trip}{}
\end{multicols}

\subsubsection*{Le type de voyage}
{\small type of travel}
\begin{multicols}{2}
    \entry{voyager seul}{}{}{to travel alone}{}
    
    \entry{voyager en famille}{}{}{to travel with family}{}
    
    \entry{voyager avec des amis}{}{}{to travel with friends}{}
    
    \entry{faire un voyage organisé}{}{}{to go on an organized trip}{}
\end{multicols}

\subsubsection*{La destination}
{\small destination}
\begin{multicols}{2}
    \entry{la plage / la mer}{}{n.f.}{beach / sea}{}
    
    \entry{la campagne}{}{n.f.}{countryside}{}
    
    \entry{la montagne}{}{n.f.}{mountains}{}
    
    \entry{l'étranger}{}{n.m.}{abroad / foreign country}{}
\end{multicols}

\subsubsection*{Les activités}
{\small activities}
\begin{multicols}{2}
    \entry{aller à la plage}{}{}{to go to the beach}{}
    
    \entry{se baigner}{}{}{to bathe / swim}{}
    
    \entry{nager}{}{}{to swim}{}
    
    \entry{se bronzer}{}{}{to tan}{}
    
    \entry{se promener}{}{}{to take a walk}{}
    
    \entry{se reposer / se détendre}{}{}{to rest / relax}{}
    
    \entry{faire la grasse matinée}{}{}{to sleep in}{}
    
    \entry{faire du tourisme}{}{}{to go sightseeing}{}
    
    \entry{faire du sport}{}{}{to do sports}{}
    
    \entry{faire de la randonnée}{}{}{to hike}{}
    
    \entry{visiter une région / une ville / un village / un musée}{}{}{to visit a region / a city / a village / a museum}{}
    
    \entry{sortir avec les amis}{}{}{to go out with friends}{}
    
    \entry{aller en boîte de nuit}{}{}{to go to a nightclub}{}
    
    \entry{danser}{}{}{to dance}{}
    
    \entry{aller au restaurant}{}{}{to go to a restaurant}{}
\end{multicols}

\subsubsection*{Les services}
{\small services}
\begin{multicols}{2}
    \entry{la location de voiture}{}{n.f.}{car rental}{}
    
    \entry{le parking}{}{n.m.}{parking}{}
    
    \entry{la piscine}{}{n.f.}{swimming pool}{}
    
    \entry{le wifi}{}{n.m.}{Wi-Fi}{}
    
    \entry{le petit déjeuner inclus}{}{n.m.}{breakfast included}{}
    
    \entry{les animaux acceptés}{}{n.m. pl.}{pets allowed}{}
    
    \entry{le service de blanchisserie}{}{n.m.}{laundry service}{}
    
    \entry{la climatisation}{}{n.f.}{air conditioning}{}
\end{multicols}

\subsubsection*{Les objets}
{\small objects}
\begin{multicols}{2}
    \entry{la valise}{}{n.f.}{suitcase}{}
    
    \entry{le passeport}{}{n.m.}{passport}{}
    
    \entry{le visa}{}{n.m.}{visa}{}
    
    \entry{la carte d'identité}{}{n.f.}{ID card}{}
    
    \entry{l'équipage de main}{}{n.m.}{carry-on luggage}{}
    
    \entry{le sac de voyage}{}{n.m.}{travel bag}{}
    
    \entry{l'appareil photo}{}{n.m.}{camera}{}
    
    \entry{la crème solaire}{}{n.f.}{sunscreen}{}
    
    \entry{le parasol}{}{n.m.}{sun umbrella}{}
    
    \entry{l'anti-moustique}{}{n.m.}{mosquito repellent}{}
    
    \entry{la casquette}{}{n.f.}{cap}{}
    
    \entry{le maillot de bain}{}{n.m.}{swimsuit}{}
    
    \entry{la serviette}{}{n.f.}{towel}{}
    
    \entry{le chargeur de téléphone}{}{n.m.}{phone charger}{}
\end{multicols}

\subsubsection*{Autres expressions}
{\small other expressions}
\begin{multicols}{2}
    \entry{prendre des vacances}{}{}{to take a vacation}{}
    
    \entry{prendre des congés}{}{}{to take time off}{}
    
    \entry{les grandes vacances}{}{n.f. pl.}{summer break / long holidays}{}
\end{multicols}

\subsection{Phrase}
\begin{enumerate}
    \item \textbf{le long de / au fil de}\footnote{along} \\
    e.g., Faire une promenade\footnote{walk} \textbf{le long de/ au fil de} la côte, de la Seine, de la rivière.
    \item \textbf{au bord de / en bord de}\footnote{by, on the edge of} \\
    e.g., Faire une promenade au bord de / en bord de la mer, de la rivière, de le lac.
    \item Parler des ses \textbf{projets}\footnote{plan}
    \begin{itemize}
        \item commencer à + infinitif $\neq$ arrêter de + infinitif
        \item avoir des projets
        \item croire que + futur, espérer que + futur
        \item penser + infinitif
    \end{itemize}
    \item \textbf{Exprimer la possibilité}: peut-être (=50\%) = probablement $\neq$ certainement/ sûrement
\end{enumerate}

\section{Grammaire}
\subsection{L'expression du temps}
\begin{enumerate}
    \item \textbf{Pour parler des jours de la semaine}:
    \begin{center}
        lundi = le prochain lundi \\
        le lundi = chaque lundi
    \end{center}
    e.g., Aujourd'hui on est lundi\footnote{Today is Monday}. J'arrive mardi. Je travaille le dimanche.
    \item \textbf{Pour parler des mois de l'année}:
    \begin{center}
        \textcolor{orange}{en} janvier = \textcolor{orange}{au} mois de janvier
    \end{center}
    e.g., Mes vacances sont au mois d'août. = Mes vacances sont en août.
    \item \textbf{Parler de la date}
    \begin{center}
        01/09 $\Rightarrow$ Le premier septembre, 12/04 $\Rightarrow$ Le douze avril 
    \end{center}
    e.g., Il est né le 15 septembre 2008.
    \item \textbf{Parler des saisons}
    \begin{center}
        \textbf{le} printemps - \textbf{l'}été - \textbf{l'}automne - \textbf{l'}hiver \\
        \textbf{au} printemps - \textbf{en} été - \textbf{en} automne - \textbf{en} hiver
    \end{center}
    e.g., C'est le printemps. Je suis né au printemps.
\end{enumerate}




\end{document}